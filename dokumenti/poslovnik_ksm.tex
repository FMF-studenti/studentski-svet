\documentclass[a4paper,oneside,12pt]{article}

\usepackage[slovene]{babel}    % slovenian language and hyphenation
\usepackage[utf8]{inputenc}    % make čšž work on input
\usepackage[T1]{fontenc}       % make čšž work on output
\usepackage[reqno]{amsmath}    % basic ams math environments and symbols
\usepackage{amssymb,amsthm}    % ams symbols and theorems
\usepackage{mathtools}         % extends ams with arrows and stuff
\usepackage{url}               % \url and \href for links
\usepackage{icomma}            % make comma a thousands separator with correct spacing
\usepackage{units}             % \unit[1]{m} and unitfrac
\usepackage{enumerate}         % enumerate style
\usepackage{array}             % mutirow
\usepackage[usenames]{color}   % colors with names
\usepackage{graphicx}          % images
\usepackage{fancyhdr}          % headers
\usepackage{titlesec}          % section settings
\usepackage{xifthen}           % if
\usepackage{needspace}         % no page break

\usepackage[bookmarks, colorlinks=true, linkcolor=black, anchorcolor=black,
  citecolor=black, filecolor=black, menucolor=black, runcolor=black,
urlcolor=black, pdfencoding=unicode]{hyperref}  % clickable references, pdf toc
\usepackage[
  paper=a4paper,
  top=2.5cm,
  bottom=2.5cm,
  textwidth=16cm,
% textheight=24cm,
]{geometry}  % page geomerty

% lists with less vertical space
\newenvironment{itemize*}{\vspace{-1.2\parskip}\begin{itemize}\setlength{\itemsep}{0pt}\setlength{\parskip}{2pt}}{\end{itemize}}
\newenvironment{enumerate*}{\vspace{-1.2\parskip}\begin{enumerate}\setlength{\itemsep}{0pt}\setlength{\parskip}{2pt}}{\end{enumerate}}
% \newenvironment{description*}{\vspace{-1.2\parskip}\begin{description}\setlength{\itemsep}{0pt}\setlength{\parskip}{2pt}}{\end{description}}

\newcommand{\Title}{Poslovnik ŠS FMF}
\newcommand{\Author}{ŠS FMF}
\title{\Title}
\author{\Author}
\date{\today}
\hypersetup{pdftitle={\Title}, pdfauthor={\Author}, pdfcreator={\Author},
pdfproducer={\Author}, pdfsubject={}, pdfkeywords={}}  % setup pdf metadata

% \pagestyle{empty}              % vse strani prazne
\setlength{\parindent}{0pt}    % zamik vsakega odstavka
\setlength{\parskip}{12pt}     % prazen prostor po odstavku
\setlength{\overfullrule}{30pt}  % oznaci predlogo vrstico z veliko črnine

% clen
\newcounter{clen}
\newenvironment{clen}[1][]{% argument je "naslov" clena
  \needspace{5\baselineskip} % reserve space -- no page breaks
  \refstepcounter{clen}
  \ifthenelse{\isempty{#1}}{
    \subsection[člen]{člen}
  }{
    \subsection[člen: (#1)]{člen}
  }
  \vspace{-\parskip}
  \ifthenelse{\isempty{#1}}{}{
    \begin{center}
      (#1)
    \end{center}
    \vspace{-\parskip}
  }
}{
  \par
}

% v nadaljenvanju
\newcommand{\vnadalj}[1]{(v nadaljevanju: #1)}

% header and footer
\pagestyle{fancy}
\fancyhf{}
\lhead{\scriptsize Študentski svet Fakultete za matematiko in fiziko \\ Jadranska 19 \\ 1000 Ljubljana}
\rhead{\scriptsize Poslovnik KŠM ŠS FMF}
\rfoot{\thepage}

% sections
\renewcommand{\thesection}{\Roman{section}}
\renewcommand{\thesubsection}{\arabic{clen}}
\titleformat*{\section}{\centering\Large\bfseries\needspace{10\baselineskip}}
\titleformat{\subsection}{\centering\bfseries\large}{\thesubsection.}{5pt}{}

% 1, 2, 3 so z besedo, ostalo s številko

\begin{document}

\vspace*{2ex}
\begin{center}
  \Huge \bfseries
  Poslovnik Komisije za študentska mnenja ŠS FMF
\end{center}
\vspace{2ex}

\section{Splošne odločbe}

\begin{clen}[pravna osnova]
  Delovanje komisije ureja Poslovnik KŠM.
  Poslovnik začne veljati 19.~aprila~2016, ko ga potrdita ŠS FMF in komisija.
\end{clen}

\begin{clen}[sestava]
  Komisija za študentska mnenja \vnadalj{komisija ali KŠM} je organ Študentskega
  sveta FMF \vnadalj{ŠS FMF}. Sestavljajo jo trije do štirje študenti Fakultete
  za matematiko in fiziko \vnadalj{FMF}, ki niso nujno člani ŠS FMF.
  Zaželeno je, da je vsak kandidat iz ene smeri študija MAT (VŠ in UN) ter FIZ
  (VŠ in UN).
\end{clen}

\begin{clen}[ustanovitev]
  Člane komisije imenuje in potrdi ŠS FMF na konstitutivni seji, ko
  imenuje tudi predsednika komisije, ki mora biti član ŠS FMF.
\end{clen}

\begin{clen}[tajnost]
  Članstvo v komisiji je tajno. Niti ŠS FMF niti komisija ne sme izdati identitete 
  članov komisije ali storiti česarkoli, kar bi lahko ogrozilo njihovo anonimnost.
\end{clen}

\begin{clen}[mandat]
  Mandat komisije traja eno leto.
\end{clen}

\begin{clen}[dolžnosti]
  Namen komisije je, da na podlagi študentskih anket in mnenj študentov oceni
  pedagoško delo predavatelja ter napiše ustrezno mnenje na prošnjo senata FMF. Če
  pedagoški delavec potrebuje mnenje za izvolitev v naziv, komisija v skladu s tem
  poslovnikom presodi, ali se z izvolitvijo strinja in praviloma izda pozitivno ali
  negativno mnenje.

  Komisija je dolžna obravnavati vse prošnje za sestavo študentskih mnenj, ki
  jih posreduje ŠS FMF in jih zaključiti v roku 60 koledarskih dni.
\end{clen}

\section{Sestava študentskih mnenj}
\begin{clen}[zbiranje podatkov]
  \label{clen:zbiranje}
  Ko komisija prejme prošnjo, začne z zbiranjem podatkov o delu pedagoškega delavca.
  Viri podatkov vključujejo:
  \begin{itemize*}
      \item študentske ankete, ki jih izvaja univerza (univerzitetne ankete iz sistema VIS)
      \item interne ankete FMF (pisne ob koncu predavanj)
      \item pretekla mnenja komisije o pedagoškem delavcu
      \item mnenja študentov, ki so v obdobju od zadnjega podanega mnenja do ponovnega poziva za mnenje imeli pouk pri tem pedagoškem delavcu
  \end{itemize*} \vspace{-1ex}
  KŠM sprejema anonimna mnenja v predalčku ŠS FMF in na elektronski naslov
  \href{mailto:studentski-svet@list.fmf.uni-lj.si}
  {\mbox{studentski-svet@list.fmf.uni-lj.si}}. V primeru malo podatkov iz anket lahko KŠM javno pozove študente za mnenje o pedagoškem delavcu. 
\end{clen}

\begin{clen}[sestava mnenja]
  Ko komisija zbere dovolj podatkov iz~\ref{clen:zbiranje}.\ člena,
  da lahko sprejme dobro utemeljeno odločitev,
  se z njimi seznanijo vsi člani komisije. 
  Komisija nato poda pisno mnenje in ga podkrepi z navedbo podatkov, ki jih je
  zbrala. Poskusi poudariti prednosti in šibke točke pedagoškega delavca z
  namenom izboljšanja kvalitete študijskega procesa.  Posebej natančno mora
  navesti razloge za negativno mnenje, da s tem spodbudi kandidata k spremembam
  v procesu dela.
\end{clen}

\begin{clen}[predlog mnenja]
  Predsednik komisije poroča ŠS FMF o predlaganih mnenjih, ki so obravnavana
  na seji ŠS FMF. Na mnenju je podpisan trenutni predsednik SŠ FMF.
\end{clen}

\section{Kriteriji za sestavo študentskega mnenja}

\begin{clen}[objektivnost in neodvisnost]
  Komisija si prizadeva, da pedagoško delo oceni  objektivno, nepristransko, ter da zastopa mnenje celotne študentske skupnosti.

  Na oceno komisije ne sme vplivati nihče, ki ni član komisije.  V nasprotnem
  primeru lahko komisija poda pritožbo ŠS FMF, ki kršitev ustrezno obravnava.
\end{clen}

\begin{clen}[ustreznost kandidatov]
  Če (bodoči) pedagoški delavec še ni opravljal pedagoškega dela,
  komisija mnenja ne poda.
\end{clen}

\vfill
Ljubljana, 19.~april~2016 \hspace*{\fill}  v imenu KŠM ŠS FMF UL \\
\hspace*{\fill} Marion van Midden \\
\hspace*{\fill}  Predsednica ŠS FMF UL

\newpage

\end{document}
% Latex template: Jure Slak, jure.slak@gmail.com
