\documentclass[a4paper,oneside,12pt]{article}

\usepackage[slovene]{babel}    % slovenian language and hyphenation
\usepackage[utf8]{inputenc}    % make čšž work on input
\usepackage[T1]{fontenc}       % make čšž work on output
\usepackage[reqno]{amsmath}    % basic ams math environments and symbols
\usepackage{amssymb,amsthm}    % ams symbols and theorems
\usepackage{mathtools}         % extends ams with arrows and stuff
\usepackage{url}               % \url and \href for links
\usepackage{icomma}            % make comma a thousands separator with correct spacing
\usepackage{units}             % \unit[1]{m} and unitfrac
\usepackage{enumerate}         % enumerate style
\usepackage{array}             % mutirow
\usepackage[usenames]{color}   % colors with names
\usepackage{graphicx}          % images
\usepackage{fancyhdr}          % headers
\usepackage{titlesec}          % section settings
\usepackage{xifthen}           % if 
\usepackage{needspace}         % no page break

\usepackage[bookmarks, colorlinks=true, linkcolor=black, anchorcolor=black,
  citecolor=black, filecolor=black, menucolor=black, runcolor=black,
  urlcolor=black, pdfencoding=unicode]{hyperref}  % clickable references, pdf toc
\usepackage[
  paper=a4paper,
  top=2.5cm,
  bottom=2.5cm,
  textwidth=16cm,
% textheight=24cm,
]{geometry}  % page geomerty

% lists with less vertical space
\newenvironment{itemize*}{\vspace{-1.2\parskip}\begin{itemize}\setlength{\itemsep}{0pt}\setlength{\parskip}{2pt}}{\end{itemize}}
\newenvironment{enumerate*}{\vspace{-1.2\parskip}\begin{enumerate}\setlength{\itemsep}{0pt}\setlength{\parskip}{2pt}}{\end{enumerate}}
\newenvironment{description*}{\vspace{-1.2\parskip}\begin{description}\setlength{\itemsep}{0pt}\setlength{\parskip}{2pt}}{\end{description}}

\newcommand{\Title}{Poslovnik ŠS FMF}
\newcommand{\Author}{ŠS FMF}
\title{\Title}
\author{\Author}
\date{\today}
\hypersetup{pdftitle={\Title}, pdfauthor={\Author}, pdfcreator={\Author},
            pdfproducer={\Author}, pdfsubject={}, pdfkeywords={}}  % setup pdf metadata

% \pagestyle{empty}              % vse strani prazne
\setlength{\parindent}{0pt}    % zamik vsakega odstavka
\setlength{\parskip}{12pt}     % prazen prostor po odstavku
\setlength{\overfullrule}{30pt}  % oznaci predlogo vrstico z veliko črnine

% clen
\newcounter{clen}
\newenvironment{clen}[1][]{% argument je "naslov" clena
  \needspace{5\baselineskip} % reserve space -- no page breaks
  \refstepcounter{clen}
  \ifthenelse{\isempty{#1}}{
    \subsection[člen]{člen}
  }{
    \subsection[člen: (#1)]{člen} 
  }
  \vspace{-\parskip}
  \ifthenelse{\isempty{#1}}{}{
    \begin{center}
      (#1)
    \end{center}
    \vspace{-\parskip}
  }
}{
  \par
}

% v nadaljenvanju
\newcommand{\vnadalj}[1]{(v nadaljevanju: #1)}

% header and footer
\pagestyle{fancy}
\fancyhf{}
\lhead{\scriptsize Študentski svet Fakultete za matematiko in fiziko \\ Jadranska 19 \\ 1000 Ljubljana}
\rhead{\scriptsize Poslovnik ŠS FMF UL}
\rfoot{\thepage}

% sections
\renewcommand{\thesection}{\Roman{section}} 
\renewcommand{\thesubsection}{\arabic{clen}} 
\titleformat*{\section}{\centering\Large\bfseries\needspace{10\baselineskip}}
\titleformat{\subsection}{\centering\bfseries\large}{\thesubsection.}{5pt}{}

% 1, 2, 3 so z besedo, ostalo s številko

\begin{document}

\vspace*{2ex}
\begin{center}
  \Huge \bfseries
  Poslovnik Študentskega sveta Fakultete za
  matematiko in fiziko Univerze v Ljubljani
\end{center}
\vspace{2ex}

\section{Splošne odločbe}

\begin{clen}[poslovnik]
Poslovnik Študentskega sveta Fakultete za matematiko in fiziko Univerze v Ljubljani
\vnadalj{Poslovnik} je najvišji akt o organiziranosti in delovanju Študentskega sveta
Fakultete za matematiko in fiziko Univerze v Ljubljani \vnadalj{Sveta}.
\end{clen}
                                                             
\begin{clen}[poslovnik ŠS UL]
V primerih, ki niso urejeni s Poslovnikom Sveta, se analogno uporablja Poslovnik
Študentskega sveta Univerze v Ljubljani \vnadalj{ŠS UL}.
\end{clen}

\begin{clen}[funckije Sveta]
\label{clen:defsvet}
Svet je najvišji študentski organ na Fakulteti za matematiko in fiziko \vnadalj{fakulteta}. Svet
zastopa interese študentk in študentov tako pri vodstvu in drugih organih fakultete ter
Univerze v Ljubljani \vnadalj{Univerza}, kot tudi nasploh. Dolžan je obveščati študente o 
zadevah, ki se tičejo študentov in Univerze, ter se do njih opredeliti. Po potrebi, kadar
gre za pomembnejše ali kompleksnejše zadeve, Svet o njih organizira javno razpravo.
\end{clen}

\begin{clen}[obveščanje]
Svet je dolžan skrbeti za prezentnost svojega dela in kakovostno komunikacijo s študenti
fakultete. To se zagotavlja preko kampanj obveščevanja študentov, spletne strani Sveta,
oglasnih desk, drugih obvestil in osebnega stika s študenti. Svet je dolžan skrbeti za
ažurnost vsaj ene oglasne deske na Oddelku za matematiko \vnadalj{OM} in vsaj ene na Oddelku
za fiziko \vnadalj{OF}. Na njih se obvešča o vsakokratni seji Sveta, z obvezno priložitvijo
predlaganega dnevnega reda, in redno objavlja vse informacije izhajajoče iz 
\ref{clen:defsvet}.~člena.
\end{clen}

\begin{clen}[žig]
Svet ima svoj žig, ki je okrogle oblike z napisom ``Študentski svet Fakultete za matematiko
in fiziko''.
\end{clen}

\begin{clen}[predstavništvo]
Svet predstavlja predsednik Sveta, v njegovi odsotnosti pa podpredsednik.
\end{clen}

\section{Konstituiranje in volitve}

\begin{clen}[razpis za kandidature in sklic volitev]
\label{clen:razpis}
Volitve se izvedejo do zaključka drugega polnega tedna v novembru na podlagi razpisa, ki
ga objavi dekan fakultete po prejemu poziva rektorja Univerze. Rok za oddajo
kandidature začne teči naslednji dan po objavi razpisa volitev in se izteče 7. dan po objavi
razpisa volitev. Volitve morajo biti sklicane v obdobju 7 delovnih dni po preteku roka za
oddajo kandidatur. Volitve ne smejo biti sklicane na ponedeljek ali petek, ravno tako ne na
dela prost dan.

Na razpis se lahko prijavijo vsi študenti fakultete s statusom študenta v tekočem
študijskem letu.

Besedilo razpisa mora vsebovati:
\begin{itemize*}
  \item obvestilo o volitvah z oznako, da gre za volitve v Svet;
  \item kraj, čas in način izvedbe volitev ter čas odprtja volišča;
  \item kratek opis nalog in pristojnosti Sveta, pravic in dolžnosti članov;
  \item način volitev Sveta;
  \item navodila za prijavo na razpis (zahtevano vsebino kandidature, rok za oddajo volilnemu
    odboru fakultete);
  \item sestavo volilnega odbora fakultete;
  \item podpis dekana.
\end{itemize*}

V njem se kandidate tudi pozove, da pred volitvami predstavijo svoj program študentom
čim več študijskih smeri na fakulteti. Besedilo razpisa mora biti objavljeno na vseh
oglasnih deskah Sveta, spletnih straneh Sveta, spletnih straneh fakultete in prek
Študentskega referata po elektronski pošti poslano vsem študentom.

Prvi delovni dan po izteku roka za vložitev kandidature volilni odbor potrdi vložene
kandidature in o njih po elektronski pošti obvesti vlagatelje. Volilni odbor zavrže
nepravočasne ali nepopolne vloge. Vloga za kandidaturo mora vsebovati pravilno
izpolnjen obrazec (mora vsebovati ime in priimek kandidata, kontaktni naslov, telefonsko
številko, datum, lastnoročni podpis kandidata), potrdilo o vpisu v tekočem študijskem letu
in seznam 20 študentov fakultete, ki kandidata podpirajo pri kandidaturi. Na seznamu
morajo biti ime, priimek, vpisna številka in podpisi podpornikov.

Volilni odbor fakultete za prejeto kandidaturo izdaja potrdila, ki jih predhodno pripravi volilni
odbor ŠS UL.

V kolikor je po preteku razpisnega roka manj kot 9 popolno oddanih vlog, predsednik
razpiše najmanj tri delovne dni dodatnega časa za oddajo kandidatur in o tem na isti način
kot ob objavi razpisa obvesti študente in dekana fakultete. Po potrebi se določi tudi nov
datum za izvedbo volitev.
\end{clen}

\begin{clen}[volitve]
\label{clen:volitve}
En delovni dan pred volitvami mora predsednik prek oglasnih desk Sveta, spletnih strani
Sveta, spletnih strani fakultete in elektronske pošte iz Študentskega referata ponovno
obvestiti študente o volitvah in jih pozvati k udeležbi. V obvestilu morajo biti vse
pomembne informacije o postopku volitev in imena kandidatov.

Volitve izvede vsaj tričlanski volilni odbor, ki ga najkasneje do 15.~oktobra imenuje volilni
odbor ŠS UL na predlog Sveta. Član volilnega odbora je lahko vsak študent fakultete s
statusom študenta. Člani volilnega odbora ne morejo biti kandidati na volitvah. Volilni
odbor izmed svojih članov imenuje predsednika volilnega odbora.

Volitve potekajo na dveh volilnih mestih, enem na OM in enem na OF. Dekan zagotovi
primerne prostore za izvedbo volitev. Volitve potekajo en dan, volilna mesta so odprta od 
9.~do 16.~ure. Ves čas volitev mora biti na vsakem volilnem mestu prisoten vsaj en član
volilnega odbora. Na vsakem volilnem mestu se nahaja po ena volilna skrinjica.

Volilni upravičenci so vsi študenti fakultete s statusom študenta. Dekan zagotovi evidenco
vpisanih študentov v tekočem študijskem letu. Volilna komisija je dolžna identificirati
vsakega volivca in preveriti ali ima status študenta. Volivec lahko voli le na oddelku, na
katerem je vpisan. Študente, ki so vpisani na obeh oddelkih, volilna komisija dodeli na eno
od volilnih mest in volivca pravilno napoti nanje. Vsak volivec lahko voli le enkrat.

Volitve so tajne. Vsak volivec lahko na volilnem listku obkroži največ 13 kandidatov. Volilni
listič je veljaven, če so kandidati obkroženi nedvoumno. Volivci oddajo volilne listke v
volilno skrinjico.

Takoj po zaprtju volilnih mest volilni odbor fakultete prešteje glasove in sprejme rezultate
volitev. Volilni odbor po tem, ko ugotovi rezultate glasovanja, pošlje glasovnice volilnemu
odboru ŠS UL. Predsednik volilnega odbora fakultete še isti dan napiše in podpiše poročilo
o volitvah ter ga odda predsedniku Sveta. Prav tako ga pošlje dekanu fakultete in
volilnemu odboru ŠS UL. Poročilo mora vsebovati opis poteka volitev z datumi in časi,
imena članov volilne komisije, ime predsednika volilne komisije, število oddanih glasov,
število veljavnih glasov, število neveljavnih glasov, rezultate volitev in seznam morebitnih
zapletov ali kršitev volilne tajnosti z natančnim časom neželenega dogodka. Predsednik
Sveta mora naslednji delovni dan po volitvah objaviti rezultate volitev na vseh oglasnih
deskah Sveta, spletnih straneh Sveta in spletnih straneh fakultete.
\end{clen}

\begin{clen}[sestava Sveta]
Svet se na podlagi rezultatov volitev sestavi izmed kandidatov na volitvah.

Vseh članov Sveta je najmanj 9 in največ 13, pri tem mora biti prednostno v Svetu vsaj po
en predstavnik naslednjih petih sklopov študijskih programov:
\begin{itemize*}
  \item kateregakoli študijskega programa 1.~bolonjske stopnje na OM;
  \item kateregakoli študijskega programa 2.~bolonjske stopnje na OM;
  \item kateregakoli študijskega programa 1.~bolonjske stopnje na OF;
  \item kateregakoli študijskega programa 2.~bolonjske stopnje na OF;
  \item enega izmed študijskih programov 3.~bolonjske stopnje, ki jih izvaja ali soizvaja
        fakulteta.
\end{itemize*}

Študent 1., 2.~ali 3.~letnika Enovitega magistrskega programa Pedagoška matematika
(2.~stopnja) se za potrebe tega poslovnika smatra za študenta 1.~bolonjske stopnje na
OM.

Svet se sestavi tako, da se najprej za vsakega od prej omenjenih petih sklopov študijskih
programov v Svet izbere kandidata ustreznega programa, ki je prejel največ glasov. Če
kandidatov za posamezen sklop ni, se ta sklop izpusti.

Preostala mesta v Svetu se zapolnijo s tistimi izmed preostalih kandidatov, ki so prejeli
največ glasov. Pri tem mora dodatno veljati, da je vsaj tretjina članov Sveta (npr.~najmanj 5
od 13) študentov na OM in vsaj tretjina članov Sveta študentov na OF. Predstavniki
študijskih programov 3.~stopnje se smatrajo kot študenti tistega oddelka, ki izvaja njihovo
izbrano smer, modul oz.~področje.
\end{clen}

\begin{clen}[konstitutivna seja]
Svet se ustanovi na konstitutivni seji. Konstitutivno sejo na podlagi poročila volilne komisije 
najkasneje teden dni po razglasitvi volilnih rezultatov skliče dekan fakultete.
Na konstitutivno sejo z elektronskim sporočilom povabi vse
izvoljene člane Sveta. S potrditvijo mandata novoizvoljenih članov na konstitutivni seji
preneha mandat dotedanjemu Svetu.

Hkrati s sklicem dekan povabi člane h kandidaturi na mesto predsednika in
podpredsednika Sveta. Rok za oddajo kandidatur je do pričetka ustanovne seje.
Kandidati se morajo na seji predstaviti s programom.

Sejo do izvolitve novega predsednika vodi dotedanji predsednik.
\end{clen}

\begin{clen}[dnevni red ustanovne seje]
Dnevni red konstitutivne seje obsega naslednje točke:
\begin{itemize*}
  \item uradni rezultati volitev;
  \item potrditev mandatov novoizvoljenih članov;
  \item predstavitev dolžnosti in pravic Sveta in njegovih članov;
  \item poročilo predsednika v prejšnjem mandatu o delu v prejšnjem mandatu z obveznim
        finančnim poročilom;
  \item predstavitev programov kandidatov ter izvolitev novega predsednika in novega
        podpredsednika ter namestnikov.
\end{itemize*}
\end{clen}

\begin{clen}[dodatne volitve]
V kolikor je članov Sveta manj kot 9 ali je bilo izvoljenih premalo članov za sklic
konstitutivne seje, je predsednik dolžan po postopku iz \ref{clen:razpis}.~in 
\ref{clen:volitve}.~člena izpeljati volitve na preostala mesta v Svetu.
\end{clen}

\section{Pravice in dolžnosti članov}

\begin{clen}[pravice članov]
Člani sveta imajo pravico in dolžnost udeleževati se sej Sveta ter sodelovati pri delu Sveta
in njegovih delovnih teles \vnadalj{DT}, katerih člani so. Član ima pravico:
\begin{itemize*}
  \item predlagati točke dnevnega reda;
  \item predlagati Svetu v sprejem sklepe;
  \item glasovati o predlogih sklepov in predlagati dopolnila teh sklepov;
  \item predlagati kandidate za vodstvo Sveta in člane delovnih teles;
  \item zahtevati pojasnila od vodstva Sveta v zvezi z delom v Svetu in v DT;
  \item razpravljati na sejah Sveta.
\end{itemize*}
\end{clen}

\begin{clen}[dolžnosti članov]
Dolžnost člana je udeležba na sejah Sveta in DT ter organov Univerze, katerih član je. Če
se seje ne more udeležiti, je svojo odsotnost dolžan vnaprej sporočiti predsedniku Sveta,
DT ali organa Univerze.

Dolžnost člana je vestno izvrševanje nalog, za katere je odgovoren.
\end{clen}

\begin{clen}[razrešitev člana]
Če se član dvakrat neupravičeno ne udeleži seje Sveta, DT ali organa Univerze, ga lahko
na naslednji seji Svet z večino vseh članov razreši. V primeru razrešitve člana, Svet
nadaljuje s svojo funkcijo, v kolikor je skupno število članov sveta vsaj 6. O razrešitvi je
osebno obveščen dekan fakultete. Na oglasno desko Sveta se objavi sklep o razrešitvi
člana ter ustrezne podrobnosti postopka.
\end{clen}

\section{Vodstvo sveta}

\begin{clen}[vodstvo]
Vodstvo Sveta predstavljajo predsednik, podpredsednik in tajnik. Za
predsednika ali podpredsednika lahko kandidira vsak član, ki vloži kandidaturo do začetka
ustanovne seje. Kandidat za predsednika ali podpredsednika je izvoljen, če zanj glasuje
večina vseh članov Sveta. Kandidata za tajnika na predlog predsednika z večino vseh
članov izvoli Svet.

Kandidat ima pravico pred začetkom volitev obrazložiti predlog kandidature
oz.~kandidaturo. Kandidat ob vložitvi kandidature ne sme biti pravnomočno 
obsojen v disciplinskem postopku

Volitve vodstva natančneje določa Poslovnik ŠS UL (93.~do vključno 102.~člen).
\end{clen}

\begin{clen}[predsednik]
Predsednik vodi Svet in je zadolžen za:
\begin{itemize*}
  \item sklicevanje in vodenje sej Sveta;
  \item izvrševanje sklepov Sveta;
  \item zagotavljanja javnosti dela Sveta;
  \item predstavljanje Sveta v ŠS UL;
  \item redno komunikacijo z vodstvom fakultete;
  \item drugo.
\end{itemize*}

Za predsednika in podpredsednika Sveta ne more biti izvoljen član upravnega odbora
katerekoli Študentske organizacije visokošolskega zavoda, član Študentskega
zbora in katerikoli nosilec funkcije na Študentski organizaciji katerekoli
univerze ali Študentski organizaciji Slovenije.
Prav tako za predsednika in podpredsednika Sveta ne more biti izvoljen član katerekoli
politične stranke ali podmladka politične stranke.

Predsednik Sveta je po funkciji član ŠS UL.
\end{clen}

\begin{clen}[podpredsednik]
Podpredsednik pomaga predsedniku in ga nadomešča v njegovi odsotnosti. Skupaj s
predsednikom je član ŠS UL.
\end{clen}

\begin{clen}[nadomestna svetnika]
Svet na ustanovni seji izmed članov z večino vseh članov izvoli nadomestna svetnika 
predsednika in podpredsednika.

Nadomestni svetnik predsednika v primeru odsotnosti predsednika opravlja
dolžnosti svetnika ŠS UL in lahko glasuje na sejah ŠS UL.

Nadomestni svetnik podpredsednika v primeru odsotnosti podpredsednika opravlja
dolžnosti svetnika ŠS UL in lahko glasuje na sejah ŠS UL.

V primeru, da se seje ne more udeležiti niti imenovani nadomestni svetnik, 
lahko predsednik ali podpredsednik pooblasti drugega člana Sveta.
\end{clen}

\begin{clen}[tajnik]
Naloge tajnika:
\begin{itemize*}
  \item pomaga predsedniku pri pripravi sej in administrativnih opravilih;
  \item piše zapisnike sej;
  \item skrbi za ažurnost oglasnih desk Sveta;
  \item vodi evidenco Sveta.
\end{itemize*}
Evidenca Sveta obsega akte Sveta, zapisnike, kandidature, arhiv Sveta, ostala gradiva.
Akti Sveta so poslovnik Sveta, vsi pravilniki Sveta in poslovniki DT Sveta.
\end{clen}

\begin{clen}[zamenjava vodstva in odstop]
Na pobudo vsaj tretjine vseh članov, lahko Svet na katerikoli seji, razen na dopisni, voli
novega predsednika, podpredsednika ali tajnika. Kandidati za mesto se morajo najprej
obvezno predstaviti članom Sveta s programom. Nov član vodstva je izvoljen,
v kolikor zanj glasuje večina vseh članov Sveta.

Vsak član vodstva ima pravico odstopiti. O tem mora pisno obvestiti člane Sveta in dekana
fakultete. Svet na naslednji seji z večino vseh članov izvoli novega člana vodstva. Član
vodstva, ki je podal izjavo o odstopu, je dolžan do izvolitve novega vodstva opravljati svoje
dolžnosti.

Po izvolitvi novega člana vodstva mora predsednik Sveta to sporočiti dekanu fakultete in
priložiti imena celotnega aktualnega vodstva. Dekan mora spremembo sporočiti v rektorat
Univerze.
\end{clen}

\section{Seje}

\begin{clen}[vrste sej]
Seje se delijo na redne, izredne in dopisne.
\end{clen}

\begin{clen}[redne seje]
Redne seje se sklicujejo mesečno. Dnevni red mora obsegati:
\begin{itemize*}
  \item sprejem zapisnika zadnje seje;
  \item poročilo predsednika o svojem delu in realizaciji sklepov;
  \item poročila predstavnikov Sveta v drugih organih;
  \item izdajo študentskih mnenj;
  \item vprašanja in pobude študentov in članov.
\end{itemize*}
\end{clen}

\begin{clen}[izredna seja]
Izredno sejo skliče predsednik v naslednjih primerih:
\begin{itemize*}
  \item na zahtevo dekana fakultete;
  \item na zahtevo vsaj tretjine članov, ki k zahtevi predložijo lastnoročne podpise;
  \item na pisno zahtevo najmanj 30 študentov fakultete;
  \item na lastno pobudo.
\end{itemize*}

V zahtevi za sklic izredne seje morajo biti navedeni razlogi za
njen sklic ter priloženo gradivo o zahtevah, o katerih naj se na njej odloča.

Predsednik je nato dolžan sklicati izredno sejo najkasneje v roku osem delovnih dni po
prejemu ustrezne zahteve. Če zahteva ni popolna, mora predsednik predlagatelja o tem
obvestiti najkasneje tri delovne dni po prejemu zahteve za sklic.
\end{clen}

\begin{clen}[dopisna seja]
Dopisna seja se skliče, če predsednik presodi, da ni mogoče zagotoviti sklepčnosti, ali
je v časovni stiski in je potrebna takojšnja odločitev Sveta. Predsednik Sveta je razloge za
sklic dolžan obrazložiti na naslednji redni seji. Dopisna seja se izvaja preko e-poštnega
seznama Sveta.

Dopisna seja traja 5 dni in je sestavljena iz:
\begin{itemize*}
  \item tri dni, namenjenih predlogom in dopolnilom članov;
  \item dva dni, namenjenih glasovanju.
\end{itemize*}
V primeru nujnosti lahko predsednik skrajša dopisno sejo na en dan, namenjen predlogom
in dopolnilom članov, ter en dan, namenjen glasovanju.
\end{clen}

\begin{clen}[javnost]
Seje so javne in se jih lahko udeleži vsak študent fakultete in vsak pedagoški delavec
fakultete. Če Svet presodi, da bi prisotnost javnosti ovirala delo Sveta, lahko s
sklepom sejo za javnost zapre.

Javnost sej se med drugim zagotavlja z objavo zapisnikov in obvestil na 
spletni strani in na oglasni deski.
\end{clen}

\begin{clen}[vabilo]
Vabilo na sejo Sveta s predlogom dnevnega reda in potrebnim gradivom pošlje predsednik
najmanj tri delovne dni vnaprej. Vabila se pošilja po elektronski pošti. Vabilo se pošlje tudi
vodstvu fakultete in vsem, katerih prisotnost zahteva dnevni red. Dnevni red se sprejema
na začetku seje in takrat so možne spremembe in dopolnitve.
\end{clen}

\begin{clen}[sklepčnost]
Svet je sklepčen, če je prisotnih večina (npr.~7 od~13) članov. Če sklepčnost pred
začetkom seje ni dosežena, se počaka 30 minut. Po preteku tega časa se seja s celotnim
dnevnim redom preloži. V primeru zagotovljene nesklepčnosti (opravičila vseh odsotnih
članov), se seja s celotnim dnevnim redom preloži takoj.
\end{clen}

\begin{clen}[sprejemanje sklepov in glasovanje]
Svet sprejema sklepe z večino glasov opredeljenih navzočih članov.

Glasovanje praviloma poteka javno z dvigom rok. Glasovanje lahko poteka tajno z glasovnicami, 
če tako zahteva vsaj en član ali je bilo tako sklenjeno pred odločanjem o
posamezni zadevi. V primeru tajnega glasovanja Svet izvoli tričlansko volilno komisijo, ki
pripravi glasovnice, jih prešteje in razglasi rezultat.

Pri glasovanju so možni glasovi ZA, PROTI in VZDRŽAN.
\end{clen}

\begin{clen}[zapisnik seje]
Pri vsaki seji se piše zapisnik, ki obsega:
\begin{itemize*}
  \item listo prisotnosti;
  \item sprejet dnevni red;
  \item sprejete sklepe;
  \item izide glasovanj;
  \item kratek povzetek diskusije.
\end{itemize*}
Zapisnik podpišeta predsednik in tajnik. Zapisnik se sprejme na naslednji seji.
\end{clen}

\begin{clen}[red na seji]
Za vzdrževanje reda na seji je zadolžen predsedujoči, ki ima pravico opomniti morebitnega
motečega navzočega in mu izreči ukrep odstranitve s seje.
\end{clen}

\begin{clen}[zaključek seje]
Po preteku dveh ur od začetka seje je predsednik dolžan dati na glasovanje sklep o
predčasnem zaključku seje. Če je sklep sprejet, se vse nezaključene točke v celoti
prenesejo na dnevni red naslednje seje.
\end{clen}

\section{Predstavniki Sveta v drugih organih}

\begin{clen}[seznam organov]
\label{clen:organi}
Svet ima svoje predstavnike v naslednjih organih:
\begin{itemize*}
  \item v Senatu fakultete (tri predstavnike);
  \item v Akademskem zboru fakultete (toliko predstavnikov, da število študentov ni manjše
          od ene petine članov Akademskega zbora fakultete  -- število se določi na začetku
          vsakega študijskega leta glede na število učiteljev, znanstvenih delavcev in
          sodelavcev);
  \item v Znanstveno pedagoškem svetu OM (enega predstavnika);
  \item v Znanstveno pedagoškem svetu OF (enega predstavnika);
  \item v ŠS UL (dva predstavnika, tj.~predsednik in podpredsednik Sveta);
  \item v Upravnem odboru Študentske organizacije FMF (enega predstavnika).
\end{itemize*}
Predstavniki v Znanstveno pedagoških svetih \vnadalj{ZPS}  OM in OF
ter v Upravnem odboru Študentske organizacije FMF \vnadalj{UO ŠO FMF} morajo biti člani
Sveta. Za predstavnika v Senatu fakultete in Akademskem zboru fakultete je lahko izvoljen
vsak študent fakultete s statusom študenta.

Svet izmed svojih članov izvoli svojega predstavnika za stik z Računalniškim
centrom fakultete \vnadalj{RC}, ki skrbi tudi za spletne strani Sveta.
\end{clen}

\begin{clen}[volitve, mandat predstavnikov, razrešitev, zamenjava in odstop]
Volitve za predstavnike v organih iz \ref{clen:organi}.~člena
se izvedejo na ustanovni ali prvi redni seji. Kandidati za vsa
mesta se morajo članom Sveta predstaviti s programom. Predsednik in podpredsednik
z izvolitvijo na svoji funkciji in po ustanovitvi ŠS UL postaneta člana ŠS UL. 
Mandat predstavnika traja eno leto oz.~do izvolitve novega predstavnika.

Predsednik Sveta je dolžan vsaj teden dni pred prvo redno sejo izvesti razpis za kandidate
za predstavnike v Senat fakultete in Akademski zbor fakultete, ki ga mora objaviti na vseh
oglasnih deskah Sveta, spletnih straneh Sveta, spletnih straneh fakultete in prek
Študentskega referata po elektronski pošti poslati vsem študentom. Rok za odajo
kandidatur mora biti dolg vsaj 5 delovnih dni.

Vloga za kandidaturo mora vsebovati pravilno izpolnjen obrazec, potrdilo o vpisu in
seznam 20 študentov fakultete, ki kandidata podpirajo pri kandidaturi. Na seznamu morajo
biti ime, priimek, vpisna številka in podpis podpornikov. V kolikor posamezen kandidat
kandidira na več predstavniških funkcij ali tudi na volitve v Svet, zadostuje en sam seznam
podpornikov za vse njegove kandidature.

V Senat fakultete in Akademski zbor fakultete so izvoljeni kandidati, ki so dobili najvišje
število glasov, pod pogojem, da je za posameznega kandidata glasovala več kot polovica
vseh članov Sveta.

Svet lahko na vsaki seji, razen na dopisni, z večino vseh članov razreši predstavnika 
katerega koli organa iz \ref{clen:organi}.~člena. 

V primeru razrešitve predstavnika v Senatu fakultete ali Akademskem zboru fakultete
je predsednik dolžan v roku 3 delovnih dni objaviti razpis za kandidature na sproščeno mesto. 

V ostalih primerih Svet glasuje o novem predstavniku. Pri tem se morajo kandidati za 
predstavnika članom predstaviti s programom. Nov predstavnik je izvoljen pod istimi 
pogoji kot prejšnji.

Vsak predstavnik Sveta v organih iz \ref{clen:organi}.~člena ima pravico odstopiti. 
O tem mora pisno obvestiti predsednika Sveta in dekana fakultete. Po odstopu Svet
po enakem postopku, kot je bil predstavnik izvoljen, izvoli novega predstavnika.

Po izvolitvi novega predstavnika mora predsednik Sveta to sporočiti dekanu.
\end{clen}

\begin{clen}[naloge predstavnikov]
Predstavniki Sveta v organih iz \ref{clen:organi}.~člena predstavljajo Svet in 
so dolžni zastopati interese in stališča Sveta ter glasovati po sklepu Sveta.
Predstavniki so dolžni pred sejo deležnega organa Svetu predstaviti relevantne 
informacije, da se lahko po potrebi skliče izredno ali dopisno sejo. O svojem 
delu sproti poročajo na vsaki redni seji in z zaključnim poročilom.
\end{clen}

\section{Delovna telesa}

\begin{clen}[delovna telesa]
Svet lahko po potrebi oblikuje stalna in nestalna delovna telesa.
Vsako stalno DT svoje delo uredi s poslovnikom, ki ga sprejme Svet z dvotretjinsko  večino.
\end{clen}

\begin{clen}[komisija za študentska mnenja]
Svet ustanovi Komisijo za študentska mnenja \vnadalj{KŠM}, ki je stalno delovno telo Sveta. KŠM
za Svet pripravlja študentska mnenja o pedagoški usposobljenosti pedagoških delavcev
fakultete. Način dela KŠM ureja \textit{Poslovnik KŠM}.

Komisija je tričlanska, od tega mora biti vsaj po en predstavnik vsakega oddelka.
Predsednik komisije mora biti član Sveta, preostala člana pa morata biti študenta fakultete.
KŠM izvoli Svet na prvi redni seji. Predsednik Sveta je dolžan vsaj 5 delovnih dni prej
objaviti razpis za kandidature.

Vloga za kandidaturo mora vsebovati pravilno izpolnjen obrazec, potrdilo o vpisu in
seznam 20 študentov fakultete, ki kandidata podpirajo pri kandidaturi. Na seznamu morajo
biti ime, priimek, vpisna številka in podpis podpornikov. V kolikor posamezen kandidat
kandidira tudi na druga študentska predstavniška mesta, zadostuje en sam seznam
podpornikov za vse njegove kandidature.

Članstvo v KŠM je tajno. Sestava KŠM je znana zgolj Svetu in KŠM.
\end{clen}

\section{Študentska mnenja}

\begin{clen}[priprava mnenj]
Način priprave študentskih mnenj ureja Pravilnik ŠS FMF o pripravi študentskih mnenj ob
izvolitvah v pedagoške nazive.
\end{clen}

\begin{clen}[sprejemanje študentskih mnenj]
Študentska mnenja se na predlog KŠM sprejemajo s sklepom.
\end{clen}

\section{Poročila}

\begin{clen}[zaključna poročila]
Zaključna poročila o svojem delu so pred zaključkom študijskega leta 
dolžni oddati in zagovarjati:
\begin{itemize*}
  \item predsednik;
  \item podpredsednik;
  \item tajnik;
  \item predstavniki v vseh organih iz \ref{clen:organi}.~člena.
\end{itemize*}
Zaključna poročila se hranijo v arhivu Sveta.
\end{clen}

\section{Prehodne in končne določbe}

\begin{clen}[pravna praznina]
V primeru, da Svet odkrije pravno praznino, lahko skladno s tem Poslovnikom oblikuje
dodatne Pravilnike Sveta.
\end{clen}

\begin{clen}[spremembe in dopolnitve]
Poslovnik in vse pravilnike ter spremembe in dopolnitve poslovnika in vseh pravilnikov
sprejema Svet z dvotretjinsko večino vseh članov Sveta.
\end{clen}

\begin{clen}[razlaga poslovnika in pravilnikov]
Za razlago poslovnika in pravilnikov je v splošnem odgovoren predsednik, med sejo pa
predsedujoči.
\end{clen}

\begin{clen}[začetek veljavnosti]
Poslovnik začne, ko ga sprejme Svet z dvotretjinsko večino. Najkasneje naslednji dan se ga objavi 
na oglasni deski Sveta.
\end{clen}

\vfill
Ljubljana, \today \hfill Marion van Midden \\
\hspace*{\fill}  Predsednica ŠS FMF UL

\newpage

\end{document}
% Latex template: Jure Slak, jure.slak@gmail.com