% !TeX spellcheck = sl_SI
\documentclass[a4paper,oneside,12pt]{article}

\usepackage[slovene]{babel}    % slovenian language and hyphenation
\usepackage[utf8]{inputenc}    % make čšž work on input
\usepackage[T1]{fontenc}       % make čšž work on output
\usepackage[reqno]{amsmath}    % basic ams math environments and symbols
\usepackage{amssymb,amsthm}    % ams symbols and theorems
\usepackage{mathtools}         % extends ams with arrows and stuff
\usepackage{url}               % \url and \href for links
\usepackage{icomma}            % make comma a thousands separator with correct spacing
\usepackage{units}             % \unit[1]{m} and unitfrac
\usepackage{enumerate}         % enumerate style
\usepackage{array}             % mutirow
\usepackage[usenames]{color}   % colors with names
\usepackage{graphicx}          % images

\usepackage[bookmarks, colorlinks=true, linkcolor=black, anchorcolor=black,
  citecolor=black, filecolor=black, menucolor=black, runcolor=black,
  urlcolor=black, pdfencoding=unicode]{hyperref}  % clickable references, pdf toc
\usepackage[
  paper=a4paper,
  top=2.5cm,
  bottom=2.5cm,
  left=2.5cm,
  right=2.5cm
% textheight=24cm,
]{geometry}  % page geomerty

\newtheorem{izrek}{Izrek}
\newtheorem{posledica}{Posledica}

\theoremstyle{definition}
\newtheorem{definicija}{Definicija}
\newtheorem{opomba}{Opomba}
\newtheorem{zgled}{Zgled}

% basic sets
\newcommand{\R}{\ensuremath{\mathbb{R}}}
\newcommand{\N}{\ensuremath{\mathbb{N}}}
\newcommand{\Z}{\ensuremath{\mathbb{Z}}}
\renewcommand{\C}{\ensuremath{\mathbb{C}}}
\newcommand{\Q}{\ensuremath{\mathbb{Q}}}
\newcommand{\T}{\ensuremath{\mathsf{T}}}

% greek letters
\let\oldphi\phi
\let\oldtheta\theta
\newcommand{\eps}{\varepsilon}
\renewcommand{\phi}{\varphi}
\renewcommand{\theta}{\vartheta}

% vektorska analiza
\newcommand{\grad}{\operatorname{grad}}
\newcommand{\rot}{\operatorname{rot}}
\renewcommand{\div}{\operatorname{div}}
\newcommand{\dpar}[2]{\frac{\partial #1}{\partial #2}}

% lists with less vertical space
\newenvironment{itemize*}{\vspace{-1.5\parskip}\begin{itemize}\setlength{\itemsep}{0pt}\setlength{\parskip}{2pt}}{\end{itemize}\vspace{-1\parskip}}
\newenvironment{enumerate*}{\vspace{-1.5\parskip}\begin{enumerate}\setlength{\itemsep}{0pt}\setlength{\parskip}{2pt}}{\end{enumerate}\vspace{-1\parskip}}
\newenvironment{description*}{\vspace{-6pt}\begin{description}\setlength{\itemsep}{0pt}\setlength{\parskip}{2pt}}{\end{description}\vspace{-1\parskip}}

\newcommand{\Title}{Pravila tekmovanja v recitiranju števila pi}
\newcommand{\Author}{Jure Slak}
\title{\Title}
\author{\Author}
\date{\today}
\hypersetup{pdftitle={\Title}, pdfauthor={\Author}, pdfcreator={\Author},
            pdfproducer={\Author}, pdfsubject={}, pdfkeywords={}}  % setup pdf metadata

\pagestyle{empty}              % vse strani prazne
\setlength{\parindent}{0pt}    % zamik vsakega odstavka
\setlength{\parskip}{10pt}     % prazen prostor po odstavku
\setlength{\overfullrule}{30pt}  % oznaci predlogo vrstico z veliko črnine

\hyphenation{ob-jav-lje-nih}

\begin{document}

\section*{Pravila tekmovanja v recitiranju decimalk števila $\boldsymbol \pi$}

\vspace{3ex}

\begin{enumerate}
  \item Tekmovalci recitirajo število $\pi$ v desetiškem sistemu, začenši s $3,14159$ \ldots
  \item Tekmovalec ima v okviru tekmovanja možnost recitirati samo enkrat.
  \item Tekmovalce se po koncu tekmovanja razvrsti padajoče po številu pravilnih decimalk, ki so jih povedali.
    Tekmovalci, ki so povedali enako število decimalk, si mesta delijo.

    Primer: \\[-20pt]
    \begin{itemize}
      \item Tekmovalec, ki pove ``3,14159'', je uspešno povedal 5 decimalk.
    \end{itemize}

  \item Pravilnost recitiranja preverja vsaj tričlanska komisija s pomočjo predhodno objavljenih
    decimalk.

  \item Tekmovalec mora decimalke recitirati dovolj razločno in počasi, da lahko komisija sproti preverja pravilnost. V primeru, da komisija ne more potrditi pravilnosti recitacije, lahko tekmovalca prekine in prosi za razločnejšo ponovitev nekaj zadnjih decimalk. Komisija se zavezuje, da se
  te možnosti poslužuje čim redkeje.

  \item Recitacija se konča, ko tekmovalec prostovoljno zaključi ali pa naredi dokončno napako
    (glej točko \ref{itm:napake}).

  \item \label{itm:napake} Če se tekmovalec zmoti, ima možnost napako popraviti, v kolikor to stori še preden  nadaljuje
    recitiranje z naslednjo decimalko. Če tega ne stori, napaka postane dokončna in komisija prekine
    recitacijo.

    Primera: \\[-20pt]
    \begin{itemize}
      \item ``3 cela 1 4 1 4, ne, narobe, 5, 9, \ldots'' \\
        Tekmovalec je napako popravil, preden je povedal naslednjo decimalko, kar
        je dovoljeno, in lahko nadaljuje z recitacijo.
      \item ``3 cela 1 4 1 4 9, ne, narobe, 5, 9, \ldots'' \\
        Tekmovalec je po napačni decimalki ``4'' povedal sicer pravilno ``9'', a napačne ``4'' pred
        tem ni popravil in zato komisija recitacijo prekine. Tekmovalec je uspešno povedal 3 decimalke.
    \end{itemize}
  \item Tekmovalec mora decimalke recitirati na pamet brez dodatnih pripomočkov.
  \item O vseh morebitnih nejasnostih odloča komisija tekmovanja.
  \item Odločitve komisije so dokončne.
\end{enumerate}

\vspace{3ex}

Pravila sprejel ŠS FMF na 6.~redni seji dne 26.~2.~2019. \\

\hspace*{\fill} predsednica ŠS FMF \\
\hspace*{\fill} Ines Meršak

\end{document}
% vim: syntax=tex
% vim: spell spelllang=sl
% vim: foldlevel=99
% Latex template: Jure Slak, jure.slak@gmail.com


