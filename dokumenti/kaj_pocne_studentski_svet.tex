\documentclass[a4paper,oneside,12pt]{article}

\usepackage[slovene]{babel}    % slovenian language and hyphenation
\usepackage[utf8]{inputenc}    % make čšž work on input
\usepackage[T1]{fontenc}       % make čšž work on output
\usepackage[reqno]{amsmath}    % basic ams math environments and symbols
\usepackage{amssymb,amsthm}    % ams symbols and theorems
\usepackage{mathtools}         % extends ams with arrows and stuff
\usepackage{url}               % \url and \href for links
\usepackage{icomma}            % make comma a thousands separator with correct spacing
\usepackage{units}             % \unit[1]{m} and unitfrac
\usepackage{enumerate}         % enumerate style
\usepackage{array}             % mutirow
\usepackage[usenames]{color}   % colors with names
\usepackage{graphicx}          % images

\usepackage[bookmarks, colorlinks=true, linkcolor=black, anchorcolor=black,
  citecolor=black, filecolor=black, menucolor=black, runcolor=black,
  urlcolor=black, pdfencoding=unicode]{hyperref}  % clickable references, pdf toc
\usepackage[
  paper=a4paper,
  top=2.5cm,
  bottom=2.5cm,
  left=2.5cm,
  right=2.5cm
% textheight=24cm,
]{geometry}  % page geomerty

\newtheorem{izrek}{Izrek}
\newtheorem{posledica}{Posledica}

\theoremstyle{definition}
\newtheorem{definicija}{Definicija}
\newtheorem{opomba}{Opomba}
\newtheorem{zgled}{Zgled}

% basic sets
\newcommand{\R}{\ensuremath{\mathbb{R}}}
\newcommand{\N}{\ensuremath{\mathbb{N}}}
\newcommand{\Z}{\ensuremath{\mathbb{Z}}}
\renewcommand{\C}{\ensuremath{\mathbb{C}}}
\newcommand{\Q}{\ensuremath{\mathbb{Q}}}
\newcommand{\T}{\ensuremath{\mathsf{T}}}

% greek letters
\let\oldphi\phi
\let\oldtheta\theta
\newcommand{\eps}{\varepsilon}
\renewcommand{\phi}{\varphi}
\renewcommand{\theta}{\vartheta}

% vektorska analiza
\newcommand{\grad}{\operatorname{grad}}
\newcommand{\rot}{\operatorname{rot}}
\renewcommand{\div}{\operatorname{div}}

% lists with less vertical space
\newenvironment{itemize*}{\vspace{-1.5\parskip}\begin{itemize}\setlength{\itemsep}{0pt}\setlength{\parskip}{2pt}}{\end{itemize}\vspace{-1\parskip}}
\newenvironment{enumerate*}{\vspace{-1.5\parskip}\begin{enumerate}\setlength{\itemsep}{0pt}\setlength{\parskip}{2pt}}{\end{enumerate}\vspace{-1\parskip}}
\newenvironment{description*}{\vspace{-6pt}\begin{description}\setlength{\itemsep}{0pt}\setlength{\parskip}{2pt}}{\end{description}\vspace{-1\parskip}}

\newcommand{\Title}{Kaj počne študentski svet?}
\newcommand{\Author}{Jure Slak}
\title{\Title}
\author{\Author}
\date{\today}
\hypersetup{pdftitle={\Title}, pdfauthor={\Author}, pdfcreator={\Author},
            pdfproducer={\Author}, pdfsubject={}, pdfkeywords={}}  % setup pdf metadata

\pagestyle{empty}              % vse strani prazne
\setlength{\parindent}{0pt}    % zamik vsakega odstavka
\setlength{\parskip}{10pt}     % prazen prostor po odstavku
\setlength{\overfullrule}{30pt}  % oznaci predlogo vrstico z veliko črnine

\begin{document}

% \thispagestyle{empty}  % ta stran prazna

\section*{Kaj dela Študentski svet?}

Študentski svet je organ študentov v okviru fakultete. Je edini organ, ki uradno predstavlja
študente znotraj fakultete in univerze. Sestavlja ga 9 do 13 študentov, ki se volijo letno,
sodelovanje pa je prostovoljno (brez plačila). Je predvsem administrativen organ, ki se ukvarja s
tekočimi študijskimi zadevami. Med najpomembnejše sodijo:
\begin{itemize}
  \item pisanje mnenj ob izvolitvah pedagoških delavcev
  \item obravnavanje predlogov in pritožb študentov glede študija ali pedagoških delavcev
\end{itemize}

\textbf{Študentski svet je na fakulteti zaradi študentov. Vsak študent se lahko na ŠS obrne s
kakršno koli težavo ali vprašanjem študijske ali socialne narave. Obravnavali ga bomo resno in
svetovali po najboljših močeh ter ga po potrebi usmerili na drug bolj primeren organ ali osebo.}

Elektronski naslov Sveta je \texttt{studentski.svet@fmf.uni-lj.si}, uradna spletna stran pa
\texttt{svet.fmf.si}.

\subsection*{Pozicije znotraj Sveta}
\begin{description*}
  \item[Predsednik ŠS FMF] skrbi za nemoteno delovanje Sveta. Vodi seje in dodeljuje tekoče naloge
    drugim članom. Prevzema odgovornost za komunikacijo z vodstvom fakultete in predstavlja
    študentski svet. Običajno pozicija letno alternira med matematiki in fiziki.
  \item[Podpredsednik ŠS FMF] nadomešča predsednika v njegovi odsotnosti in mu pomaga pri tekočih
    zadevah. Običajno je iz nasprotnega oddelka kot predsednik in skrbi za zadeve, ki se tičejo
    njegovega oddelka.
  \item[Tajnik ŠS FMF] zapisuje dogajanje med sejami in pripravi zapisnik sej Sveta. Skrbi za
    natančen povzetek sej sveta in njihovo objavo ter posredno za dobro obveščenost študentov.
  \item[Predstavnik za stike z RC] skrbi za spletno stran in elektronski naslov Sveta ter ostalo
    informacijsko podporo s pomočjo RC FMF.
  \item[Komisija za študentska mnenja] piše študentska mnenja o profesorjih in asistentih ob
    izvolitvah s pomočjo študentskih anket. Za uspešno izvolitev mora pedagoški delavec dobiti
    pozitivno študentsko mnenje. V primeru negativnega mnenja ŠS z vodstvom fakultete in
    obravnavanim kandidatom izvede sestanek in sprejme ukrepe za njegovo izboljšanje. Članstvo v komisiji je tajno.
  \item[Predstavniki letnikov] so študenti vseh letnikov vsakega programa na fakulteti. Vsak letnik
    izbere svojega predstavnika in ga prek vodstva fakultete posreduje ŠS FMF. ŠS preko
    predstavnikov letnikov sporoča pomembne informacije vsem študentov ali pa dobiva informacije o
    posameznih študijskih programih ali pa se pozanima o kvaliteti izvajanja predavanj ali vaj pri
    določenem profesorju ali asistentu.
\end{description*}

\subsection*{Predstavniki v organih FMF}
\begin{description*}
  \item[Senat FMF] je najvišji organ fakultete in ima zaključne odločitve pri tekočih izvolitvah,
    spremembah programov in ostalih tekočih zadevah. ŠS ima v senatu 3 predstavnike. Senat se
    sestaja mesečno.
  \item[Upravni odbor FMF] se odloča o finančnih zadevah fakultete, med drugim o cenah storitev, ki
    vplivajo na študente. ŠS ima v UO 1 predstavnika.
  \item[Znanstevno pedagoški svet] na vsakem oddelku posebej odloča o raziskovalnih in pedagoških
    zadevah, npr.~izvolitvah, spremembah programov ipd. Sestaja se mesečno, odločitve se posredujejo
    senatu FMF.  ŠS ima v vsakem ZPS po 1 predstavnika.
  \item[Akademski zbor] je zbor vseh pedagoških delavcev, ki se sestanejo enkrat letno in dajejo
    predloge in pobude senatu. Prav tako se izraža podporo kandidatom za dekana ob volitvah. ŠS ima
    v akademskem zboru 20 predstavnikov.
  \item[Disciplinska komisija] se sestane kadar pride do goljufij pri preverjanju znanja ali pri
    drugih kršitvah pravilnika s strani študentov. ŠS ima v disciplinski komisiji 2 predstavnika.
  \item[Komisija za etična vprašanja] se sestane v primeru etičnih vprašanj glede raziskovalnega ali
    pedagoškega dela. ŠS ima v njej enega predstavnika.
  \item[Komisija za kakovost] se sestane skupaj z Nacionalno agencijo RS za kakovost v visokem
    šolstvu ob ponovni akreditaciji študijskih programov. Običajno se poleg članov Sveta sestanejo
    tudi s študenti študijskih programov v obravnavi.
  \item[Komisija za samoevaluacijo] pripravlja notranje poročilo o kakovosti fakultete in njenega
    delovanja. ŠS ima v njej 2 predstavnika, sestanki pa so zelo redki.
\end{description*}

\end{document}
% vim: syntax=tex
% vim: spell spelllang=sl
% vim: foldlevel=99
% Latex template: Jure Slak, jure.slak@gmail.com
