\documentclass[oneside, 12pt]{article}

\usepackage[slovene]{babel}    % slovenian language and hyphenation
\usepackage[utf8]{inputenc}    % make čšž work on input
\usepackage[T1]{fontenc}       % make čšž work on output
\usepackage[reqno]{amsmath}    % basic ams math environments and symbols
\usepackage{amssymb,amsthm}    % ams symbols and theorems
\usepackage{mathtools}         % extends ams with arrows and stuff
\usepackage{url}               % \url and \href for links
\usepackage{icomma}            % make comma a thousands separator with correct spacing
\usepackage{units}             % \unit[1]{m} and unitfrac
\usepackage{enumerate}         % enumerate style
\usepackage{array}             % mutirow
\usepackage[usenames]{color}   % colors with names
\usepackage{graphicx}          % images

\usepackage[bookmarks, colorlinks=true, linkcolor=black, anchorcolor=black,
  citecolor=black, filecolor=black, menucolor=black, runcolor=black,
  urlcolor=black, pdfencoding=unicode]{hyperref}  % clickable references, pdf toc
\usepackage[
  paper=a3paper,
  top=2cm,
  bottom=2cm,
  left=2cm,
  right=2cm,
%   textwidth=19cm,
% textheight=24cm,
]{geometry}  % page geomerty

\newtheorem{izrek}{Izrek}
\newtheorem{posledica}{Posledica}

\theoremstyle{definition}
\newtheorem{definicija}{Definicija}
\newtheorem{opomba}{Opomba}
\newtheorem{zgled}{Zgled}

% basic sets
\newcommand{\R}{\ensuremath{\mathbb{R}}}
\newcommand{\N}{\ensuremath{\mathbb{N}}}
\newcommand{\Z}{\ensuremath{\mathbb{Z}}}
\renewcommand{\C}{\ensuremath{\mathbb{C}}}
\newcommand{\Q}{\ensuremath{\mathbb{Q}}}

% lists with less vertical space
\newenvironment{itemize*}{\vspace{-6pt}\begin{itemize}\setlength{\itemsep}{0pt}\setlength{\parskip}{2pt}}{\end{itemize}}
\newenvironment{enumerate*}{\vspace{-6pt}\begin{enumerate}\setlength{\itemsep}{0pt}\setlength{\parskip}{2pt}}{\end{enumerate}}
\newenvironment{description*}{\vspace{-6pt}\begin{description}\setlength{\itemsep}{0pt}\setlength{\parskip}{2pt}}{\end{description}}

\newcommand{\Title}{Ankete}
\newcommand{\Author}{Jure Slak}
\title{\Title}
\author{\Author}
\date{\today}
\hypersetup{pdftitle={\Title}, pdfauthor={\Author}, pdfcreator={\Author},
            pdfproducer={\Author}, pdfsubject={}, pdfkeywords={}}  % setup pdf metadata

\pagestyle{empty}              % vse strani prazne
\setlength{\parindent}{0pt}    % zamik vsakega odstavka
\setlength{\parskip}{8pt}     % prazen prostor po odstavku
\setlength{\overfullrule}{30pt}  % oznaci predlogo vrstico z veliko črnine

\newcommand{\veliko}[1]{\scalebox{3}{\textbf{#1}}}
\newcommand{\srednje}[1]{\scalebox{2.4}{#1}}
\newcommand{\povezava}[1]{\begin{center}\vspace{-20pt}\Large \url{#1}\end{center}}

\begin{document}

\Huge

\veliko{Kaj}\srednje{ se dogaja?}

Uvaja se nov sistem anket na Univerzi v Ljubljani. \\
Interne ankete na FMF ostajajo enake kot vedno.

\vspace{12pt}
\veliko{Kakšne}\srednje{ so nove ankete?}

Ankete so krajše in razdeljene na dva dela. Vsak semester dobijo študenti dve
seriji anket. Prvo serijo moraš rešiti pred prvo prijavo na izpit (ali
pred vpisom ocene, če se na izpite ne prijavljate), sicer se ne moraš prijaviti,
drugo pa, ko si predmet opravil.

\vspace{12pt}
\veliko{Kje}\srednje{ jih lahko rešim?}

Rešiš jih na sistemu VIS, pod zavihkom
\begin{center}
\vspace{-16pt}
\framebox{\texttt{Ankete}\rule[-5pt]{0pt}{0pt}} / \framebox{\texttt{Izpolnjevanje anket UL}}
\end{center}
\vspace{-12pt}
Preusmerjen boš na Arnes strežnik, kjer rešiš anketo.

\vspace{12pt}
\veliko{Kdaj}\srednje{ jih moram rešiti?}

Prvo serijo anket lahko rešiš od \textbf{15.~12.~2015}, torej že sedaj.
Priporočamo, da to storiš čim prej, da ne boš imel težav s prijavami na izpit
ali vpisi ocen! Prva serija za drugi semester se odpre \textbf{xx.~xx.~2016}.

\vspace{12pt}
\veliko{Zakaj}\srednje{ so pomembne?}

Glavni namen je izboljšanje kvalitete študija.
Vpogled v ankete dobijo:
\vspace{-12pt}
\begin{itemize*}
  \item \textbf{pedagoški delavec}, o katerem je anketa, z namenom, da vidi povratno mnenje
    študentov in po potrebi izboljša svoje delo,
  \item \textbf{skrbnik programa}, da ugotovi kaj je na programu dobrega in slabega, in
    po potrebi sproži postopke za spremembe,
  \item \textbf{študentski svet}, za namene podajanja mnenj o pedagoških delavcih ob
    izvolitvah v naziv,
  \item \textbf{dekan}, ki se je dolžan pogovoriti z 10\% najslabše ocenjenimi.
\end{itemize*}

Za več informacij vprašaj na \texttt{forum.fmf.si/c/studentski-organi} ali
\texttt{studentski.svet@fmf.uni-lj.si}.

% text here

\end{document}
% vim: syntax=tex
% vim: spell spelllang=sl
% vim: foldlevel=99
% Latex template: Jure Slak, jure.slak@gmail.com

