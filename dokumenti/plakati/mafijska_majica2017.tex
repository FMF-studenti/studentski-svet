\documentclass[a4paper,oneside,12pt]{article}

\usepackage[slovene]{babel}    % slovenian language and hyphenation
\usepackage[utf8]{inputenc}    % make čšž work on input
\usepackage[T1]{fontenc}       % make čšž work on output
\usepackage[reqno]{amsmath}    % basic ams math environments and symbols
\usepackage{amssymb,amsthm}    % ams symbols and theorems
%\usepackage{mathtools}         % extends ams with arrows and stuff
\usepackage{url}               % \url and \href for links
\usepackage{icomma}            % make comma a thousands separator with correct spacing
\usepackage{units}             % \unit[1]{m} and unitfrac
\usepackage{enumerate}         % enumerate style
\usepackage{array}             % mutirow
% \usepackage[usenames]{color}   % colors with names
\usepackage{graphicx}          % images

\usepackage[bookmarks, colorlinks=true, linkcolor=black, anchorcolor=black,
  citecolor=black, filecolor=black, menucolor=black, runcolor=black,
  urlcolor=black, pdfencoding=unicode]{hyperref}  % clickable references, pdf toc
\usepackage[
  paper=a4paper,
  top=1.5cm,
  bottom=1.5cm,
  textwidth=19cm,
% textheight=24cm,
]{geometry}  % page geomerty

\newtheorem{izrek}{Izrek}
\newtheorem{posledica}{Posledica}

\theoremstyle{definition}
\newtheorem{definicija}{Definicija}
\newtheorem{opomba}{Opomba}
\newtheorem{zgled}{Zgled}

% basic sets
\newcommand{\R}{\ensuremath{\mathbb{R}}}
\newcommand{\N}{\ensuremath{\mathbb{N}}}
\newcommand{\Z}{\ensuremath{\mathbb{Z}}}
\renewcommand{\C}{\ensuremath{\mathbb{C}}}
\newcommand{\Q}{\ensuremath{\mathbb{Q}}}

% lists with less vertical space
\newenvironment{itemize*}{\vspace{-6pt}\begin{itemize}\setlength{\itemsep}{0pt}\setlength{\parskip}{2pt}}{\end{itemize}}
\newenvironment{enumerate*}{\vspace{-6pt}\begin{enumerate}\setlength{\itemsep}{0pt}\setlength{\parskip}{2pt}}{\end{enumerate}}
\newenvironment{description*}{\vspace{-6pt}\begin{description}\setlength{\itemsep}{0pt}\setlength{\parskip}{2pt}}{\end{description}}

\newcommand{\Title}{Natečaj za dizajn majice za mafijski piknik}
\newcommand{\Author}{Jure Slak}
\title{\Title}
\author{\Author}
\date{\today}
\hypersetup{pdftitle={\Title}, pdfauthor={\Author}, pdfcreator={\Author},
            pdfproducer={\Author}, pdfsubject={}, pdfkeywords={}}  % setup pdf metadata

\pagestyle{empty}              % vse strani prazne
\setlength{\parindent}{0pt}    % zamik vsakega odstavka
\setlength{\parskip}{30pt}     % prazen prostor po odstavku

\newcommand{\veliko}[1]{\scalebox{4}{#1}}
\newcommand{\srednje}[1]{\scalebox{2.5}{#1}}
\newcommand{\povezava}[1]{\begin{center}\vspace{-20pt}\Large \url{#1}\end{center}}

%\usepackage{pst-barcode}
%\usepackage{auto-pst-pdf} % uncomment this if used with pdflatex
\usepackage{fancyhdr}

\begin{document}


\begin{center}
  \srednje{ŠS FMF razpisuje natečaj}

  \setlength{\baselineskip}{40pt}
  \vspace{-35pt}
  \srednje{za} \\
  \veliko{\bf dizajn majice} \\
  \srednje{za} \\[10pt]
  \veliko{\bf Mafijski piknik 2017!}

  \vspace{-15pt}
  \includegraphics[width=0.4\textwidth]{majica.png}
\end{center}

\fontsize{24}{26}\selectfont

\vspace{-65pt}
Zaželeno je, da:
\vspace{-35pt}
\begin{itemize}
  \item sta motiva spredaj in zadaj
  \item je motiv primeren za oba oddelka fakultete
  \item so napisi v slovenščini, grafika pa enobarvna
  \item dizajn oddaš v vektorski obliki do \textbf{21.~aprila~2017} \\[5pt]
        na \texttt{studentski.svet@fmf.uni-lj.si}.
\end{itemize}
\vspace{-25pt}
Predlagaš lahko tudi barvo majice, za nagrado pa dobiš čast in slavo ter karto za piknik.

\fontsize{10}{11}\selectfont
Opomba za bruce: Mafijski piknik je vsakoletno tradicionalno srečanje, ki se ga
udeležijo študenti in zaposleni FMF, kjer se j\'{e}, pije in druži dolgo v noč.
Običajno poteka proti koncu maja (letos predvidoma 17.). Z nakupom karte dobiš
majico z nekim mafijskim motivom (zato natečaj) ter ``neomejeno'' hrano in
pijačo za cel večer. Spodobi se, da se ga vsaj enkrat udeleži vsak pravi
mafijec.

\end{document}
% vim: syntax=tex
% vim: spell spelllang=sl
% vim: foldlevel=99
% Latex template: Jure Slak, jure.slak@gmail.com

