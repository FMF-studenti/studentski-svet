\documentclass{article}
\usepackage[utf8]{inputenc}
\usepackage[
  paper=a4paper,
  top=2.5cm,
  bottom=2.5cm,
  textwidth=16cm,
% textheight=24cm,
]{geometry}  % page geomerty

\title{SŠ FMF 2015/16}
\author{Jure Slak}
\date{\today}

\setlength{\parindent}{0pt}
\setlength{\parskip}{10pt}

\newcounter{vpr}
\newcommand{\vpr}[1]{\refstepcounter{vpr}\textbf{Vprašanje \arabic{vpr}.} \emph{#1}\\}
\newcommand{\odg}[1]{\textbf{Odgovor \arabic{vpr}.} #1}

\begin{document}

\begin{center}
  \Huge \bf Kviz
\end{center}

\vpr{Koliko je stopnic od sredine medbrezja do vrha stavbe matematike (5.\ nadstropje)?}
\odg{146 (Veno)}

\vpr{Koliko je stopnic od sredine medbrezja do vrha stavbe fizike?}
\odg{142 (Veno)}

\vpr{Koliko brez je na parkirišču med matematiko in fiziko?}
\odg{68 = 3*12 na sredini + 32 okrog (Veno)}

\vpr{Koliko dni je stara najstarejša vratarica na matematiki?}
\odg{20498, rojena 4.4.1960 (Jure)}

\vpr{Koliko knjiznicnih enot je v knjižnici za mehaniko?}
\odg{5.385 (Jure cobiss)}

\vpr{Koliko flomastrov porabimo na matematiki na teden?}
\odg{200 (po oceni vratarice)}

\vpr{Koliko mesecev smo matematiki že v novi stavbi?}
\odg{Od oktobra 2006, torej 116 mesecev}

\vpr{Skozi koliko oken na fiziki se lahko smeješ matematikom (prizidek ne šteje)?}
\odg{183 (Jure)}

\vpr{Koliko kavnih avtomatov za študente je skupno v stavbah FMF-ja?}
\odg{3, 2.05, Fizika, VFP}

\vpr{Koliko je kletk v četrtem nadstropju matematike?}
\odg{8, od tega 5 za študente (Jure)}

\vpr{Koliko Radlerjev smo kupili za ta Mafijski piknik?}
\odg{TODO}

\vpr{Koliko je prišel račun iz Merkatorja za Mafijski piknik (hrana in brezalkoholna pijača)?}
\odg{475,43 (Veno)}

\vpr{Koliko študentov je lani redno (tj.\ v 3 letih) diplomiralo na programu matematika?}
\odg{6 (Jure)}

\vpr{Koliko mnenj o zaposlenih na FMF je v letošnjem šolskem letu izdal ŠS FMF?}
\odg{34 (Jure)}

\end{document}
