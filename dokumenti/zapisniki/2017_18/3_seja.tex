
\documentclass{seja}

\newcommand{\SejaNum}{3}
\newcommand{\Title}{Zapisnik \SejaNum.~redne seje Študentskega sveta FMF}
\newcommand{\DateSeja}{11.~1.~2018}
\newcommand{\DateZapisnik}{31.~1.~2018}

\begin{document}

Prisotni člani in članice ŠS FMF UL:
Neža Dimec,
Klemen Ducman,
Matej Janežič,
Pia Klemenc,
Ana Kregar,
Ines Meršak,
Klementina Pirc,
Jure Slak,
Miha Šifrer,
Teo Trifkovič.

Opravičeno odsotni člani in članice ŠS FMF:
Aljaž Bavec, Katarina Šipec.


\begin{red*}
  \item Sprejem zapisnika prejšnje seje
  \item Poročilo predsedstva
  \item Oblikovanje delovnega telesa za prenovo Poslovnika ŠS FMF
  \item Poročila predstavnikov v organih in komisijah
  \item Sprejem študentskih mnenj
  \item Poročila organizatorjev stojnice in brucovanja
  \item Financiranje tekmovanj
  \item Hoodie
  \item Pi-dan
  \item Družabni večeri
  \item ŠSUL-kandidature
\end{red*}

\begin{ad}
	
	\item
	ŠS FMF soglasno potrjuje zapisnik 2. seje ŠS FMF UL.
	
	\item 
	Klemen Ducman poroča o dogajanju v ŠS UL.
	
	\item
	Ustvari se delovno telo za izdelavo novega poslovnika ŠS FMF: \\
	Klemen Ducman (vodja),  Ana Kregar, Ines Meršak. \\
	ŠS FMF za pomoč prosi predstavnika iz vodstva fakultete.
	
	\item
	\textbf{Predstavnik ZPS Oddelka za matematiko poroča o dveh sejah:}\\
	decembrska seja:
		\begin{itemize}
			\item 
			obravnava in popravki novega pravilnika FMF
		\end{itemize}
	januarska seja:
		\begin{itemize}
			\item 
			obravnava upadlega vpisa na podiplomski študij FiMa
			\item 
			obravnava spremembe programa IŠRM
			\item
			program PEM dobi akreditacijo za nedoločen čas
			\item
			za program FiMa se ponuja predmet Programiranje 1 kot izbirni predmet
			\item
			za ponavljanje 1.~letnika FiMa potrebno poleg 30 kreditnih točk se ANA1 ali ALG1
			\item 
			zaradi varstva osebnih podatkov interne ankete niso več potrebne
			\item 
			obravnava želje študentov podiplomskega študija FiMa, da bi imeli ločene vaje od MAT za nekatere predmete						
		\end{itemize}
	
	\textbf{Poročilo predstavnice v komisiji za kakovost:}\\
	Obravnava naslednjih tematik za letno poročilo:
		\begin{itemize}
			\item 
			obravnava obvestil na Facebook in spletni strani
			\item 
			obravnava obštudijskih dejavnosti in storitev za študente (dogodki v Mafiji, dosežki na tekmovanjih, bralni klub, itd.)
		\end{itemize}
	
	\item
	Sprejem študentskih mnenj:
	\begin{sklep}
		ŠS FMF pregleda in sprejme študentska mnenja za naslednje pedagoške delavce:
		\begin{itemize}
			\item Žiga Kos
			\item Tilen Marc
			\item Andrej Mihelič
			\item Matej Praprotnik
			\item Gregor Skačej
			\item Damijan Škrk
			\item Matjaž Vencelj
			\item Mojca Vilfan
		\end{itemize}	
	\end{sklep}
	
	\begin{sklep}
    	ŠS FMF zaradi pomanjkanja informacij ne bo izdal mnenj za naslednje pedagoške delavce:
    	\begin{itemize}
    		\item Boštjan Drolc
    	\end{itemize}	
	\end{sklep}
	
	\item 
    Organizacija božične stojnice in brucovanja uspešno izvedena.

	\item
	ŠS FMF obravna prošnjo za sofinanciranje tekmovanja IPT. //
    S tekmovanjem smo se seznanili, vabimo pa morebitne udeležence, da predstavijo svoje projekte, saj jih ŠS FMF morda lahko sofinancira. 


	\item
	ŠS FMF oblikuje delovno skupino za organizacijo izdelave in prodaje FMF Hoodijev: \\
	Matej Janežič (vodja), Klemen Ducman, Pia Klemenc, Ines Meršak, Klementina Pirc, Jure Slak, Katarina Šipec, Teo Trifkovič.

	\item	
	ŠS FMF oblikuje delovno skupino za organizacijo Pi-dneva: \\
	Klemen Ducman (vodja), Matej Janežič, Pia Klemenc, Klementina Pirc, Jure Slak, Miha Šifrer.

	\item
	ŠS FMF oblikuje delovno skupino za organizacijo družabnih večerov: \\
	Ines Meršak (vodja), Matej Janežič, Pia Klemenc, Katarina Šipec.
	
	\item 
	Klemen Ducman in Ines Meršak predstavita možne kandidature v ŠS UL.
	
	\item
	Razno:
		\begin{itemize}
			\item
				\begin{sklep}
				    ŠS FMF se strinja, da bo financiral naslednjih 5 računalniških delavnic, skupaj v višini do 250 EUR.
				\end{sklep}	
			\item
			tabla v kletkah še vedno ni popravljena
			\item 
			ŠO porabil sredstva za nakup čajev za čajni kotiček
			\item 
			ŠS FMF diskutira o promociji Google Hash Code tekmovanja
			\item 
			ŠS FMF oblikuje delovno skupino za organizacijo delovnega vikenda: \\
			Teo Trifkovič (vodja), Pia Klemenc, Matej Janežič, Ines Meršak
			\item 
			ŠS FMF ponovno obravnava problematiko Moderne fizike, kjer ni spremembe
		\end{itemize} 	
	
\end{ad}

\end{document}
