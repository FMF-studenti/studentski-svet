
\documentclass{seja}

\newcommand{\SejaNum}{4}
\newcommand{\Title}{Zapisnik \SejaNum.~redne seje Študentskega sveta FMF}
\newcommand{\DateSeja}{7.~3.~2018}
\newcommand{\DateZapisnik}{8.~3.~2018}

\begin{document}

Prisotni člani in članice ŠS FMF UL:
Aljaž Bavec,
Neža Dimec,
Matej Janežič,
Pia Klemenc,
Ines Meršak,
Klementina Pirc,
Jure Slak,
Miha Šifrer,
Katarina Šipec,
Teo Trifkovič.
Opravičeno odsotni člani in članice ŠS FMF UL:
Klemen Ducman, Ana Kregar.



\begin{red*}
  \item Sprejem zapisnika prejšnje seje
  \item Poročilo predsedstva
  \item Poročila predstavnikov v organih in komisijah
  \item Sprejem Študentskih mnenj
  \item Pi-dan
  \item Družabni večer
  \item Hoodie
  \item Rokometna ekipa
  \item Poročilo o organizaciji tekmovanja Google Hash Code
  \item Izvolitev predstavnika v Komisijo za študijske zadeve
\end{red*}

\begin{ad}
	
	\item
	ŠS FMF soglasno potrjuje zapisnik 3. seje ŠS FMF UL.
	
	\item 
	\textbf{Ines Meršak poroča o sestanku s predstavniki letnikov in informativnem dnevu:}
	 \begin{itemize}
	 	\item 
	 	pogovori z vodstvom o problematikah posameznih letnikov
	 	\item 
	 	obravnava Erazmus programa, predlagana rešitev je, da bi se izbralo katere predmete se ponuja Erazmus študentom v angleščini
	 	\item 
	 	predstavniki letnikov fizike še niso izbrani
	 \end{itemize}	
	Informativni dan je bil dobro organiziran in imel visoko udeležbo.
	
	
	\item
	\textbf{Predstavnica v senatu FMF poroča o zadnji seji senata FMF:}
	\begin{itemize}
		\item 
		obravnava se predlog, da bi ŠS FMF imel ves čas dostop do anket, saj bi to olajšalo delo komisije za kakovost
		\item 
		doktorski študij se podaljša iz treh na štiri leta v študijskem letu 2019/20
	\end{itemize}
	\textbf{Predstavnik ZPS Oddelka za matematiko poroča o februarski seji:}
	\begin{itemize}
		\item 
		obravnava se sprememba doktorskega študija
		\item 
		obravnava se problematika Erazmus študija
		\item 
		knjižnica FMF bo organizirala ureditev in omogočila dostopnost za e-gradiva kot npr. skripte
	\end{itemize}

	\item
		Sprejem študentskih mnenj:
	\begin{sklep}
		ŠS FMF pregleda in sprejme študentska mnenja za naslednje pedagoške delavce:
		\begin{itemize}
			\item Dean Cvetko
			\item Damir Franetič
			\item Stanislav Vrtnik
		\end{itemize}	
	\end{sklep}
	\begin{sklep}
    	ŠS FMF zaradi pomanjkanja informacij ne bo izdal mnenj za naslednje pedagoške delavce:
    	\begin{itemize}
    		\item Žiga Štancar
    	\end{itemize}	
	\end{sklep}
 
	\item
	\begin{sklep*}
	ŠS FMF za predstavnika v Komisiji za študijske zadeve imenuje Mateja Janežiča, za njegovega namestnika pa Aljaža Bavca.
	\end{sklep*}
	
	\item 
	Za Pi-dan se sestavi obrazec za prijavo. ŠS FMF bo sofinanciral nagrade v višini 150 EUR iz denarnih sredstev za tekmovanja.
	Člani komisije so:
	Pia Klemenc, Klementina Pirc, Miha Šifrer, Katarina Šipec.
	O tekmovanju se sporoči predstavnikom letnikov preko e-maila.
	
	\item	
	Prvi družabni večer bo 8.~marca v Mafiji. O dogodku se se sporoči predstavnikom letnikov preko e-maila.
	
	\item 
	Ekipa za hoodie je pripravila naročilne obrazce in se pozanimala o ceni puloverjev in tiskov. Trenutna predvidena cena znaša 25 EUR.
	
	\item 
	V primeru sestavitve rokometne ekipe za tekmovanje bo ŠS FMF poskrbel, da bo vodstvo fakultete poskrbelo za drese.
	
	\item
	Na Google Hash Code tekmovanju je sodelovalo 70 ljudi. Tekmovanje je potekalo brez težav. Člani najboljše slovenske ekipe so študenti FMF in so se uvrstili na 59.~mesto.
	
	\item
	ŠS FMF sofinancira UPM tekmovanje v višini 200 EUR.
	
\end{ad}

% spremeni footer
\makeatletter \global\let\@enddocumenthook\@empty \makeatother
\AtEndDocument{
Zapisal Matej Janežič, pregledala Ines Meršak.

\begin{flushright}
  Podpredsednica ŠS FMF \\
  Ines Meršak
\end{flushright}}

\end{document}
