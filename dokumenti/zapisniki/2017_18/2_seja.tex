
\documentclass{seja}

\newcommand{\SejaNum}{2}
\newcommand{\Title}{Zapisnik \SejaNum.~redne seje Študentskega sveta FMF}
\newcommand{\DateSeja}{6.~12.~2017}
\newcommand{\DateZapisnik}{6.~12.~2017}

\begin{document}

Prisotni člani in članice ŠS FMF UL:
Aljaž Bavec,
Neža Dimec,
Klemen Ducman,
Pia Klemenc,
Ines Meršak,
Klementina Pirc,
Jure Slak,
Miha Šifrer,
Katarina Šipec,
Teo Trifkovič.

Opravičeno odsotni člani in članice ŠS FMF UL:
Matej Janežič, Ana Kregar.


\begin{red*}
  \item Sprejem zapisnika prejšnje seje
  \item Izvolitev namestnika tajnika
  \item Določitev predstavnika za preizkusno predavanja dr.~Martina Hrovata
  \item Poročilo predsedstva
  \item Poročila predstavnikov v organih in komisijah
  \item Poročila organizatorjev stojnice in brucovanja
\end{red*}

\begin{ad}

\item
  ŠS FMF soglasno potrjuje zapisnik 1.~(konstitutivne) seje ŠS FMF UL.

\item 
  Za namestnico tajnika kandidira Neža Dimec. \\
  Rezultat glasovanja: soglasno potrjena.
  
  \begin{sklep*}
    ŠS FMF imenuje Nežo Dimec za namestnico tajnika.
  \end{sklep*}
  
\item
  \begin{sklep}
    ŠS FMF za predstavnika za preizkusno predavanje dr.~Martina Hrovata določi Klemna Ducmana.
  \end{sklep}
  Sklep soglasno sprejet.
  
\item
  Na seji upravnega odbora smo prejeli novi finančni načrt za ŠS FMF UL. Ta je takšen, da je precej strogo določeno in omejeno, kako se sredstva lahko porabljajo. Ukinil se bo čajni kotiček, saj nimamo več sredstev, ki bi bila temu namenjena. Rešitve zaenkrat še ni.
  
  Iz sredstev bomo lahko pridobili financiranje za izvedbo $\pi$-dneva.
  
  Finančna organizacija mafijskega piknika bo potekala v upanju na ustrezno sodelovanje s strani Mafije.
  
  \textit{Seji se pridruži Teo Trifkovič.}
  
\item
  Ines Meršak kratko poroča o srečanju na Senatu FMF: novosti na področju doktorskega študija, potrjenih je nekaj študijskih programov.
  
  \textit{Sejo zapusti Miha Šifrer in odide na sejo ZPS.}
  
  Klementina Pirc kratko poroča o sestanku Komisije za kakovost: izvedena je bila delavnica o pisanju poročil.
  
\item
  Klemen Ducman predlaga oblikovanje delovnih skupin za organizacijo dogodkov. Delovna skupina se lahko sestaja izven sej ŠS FMF UL. To pomeni, da se organizacija dogodkov odvija na srečanjih delovne skupine in ne na sejah sveta, razen kadar je potrebna določena obravnava tudi s strani sveta.
  
  Organizacija stojnice:
  \begin{itemize}
      \item Pripravljen je nakupovalni seznam.
      \item Pripravljen je urnik, kdaj bo kdo sodeloval na stojnici.
      \item Delno financiranje s strani ŠO FMF, čakamo na naročilnico.
  \end{itemize}
  
  Organizacija brucovanja:
  \begin{itemize}
      \item Kratko poročanje o preverjenih klubih oz. prostorih.
      \item Zadnji teden v decembru je večinoma povsod zapolnjen, saj so takrat že božične zabave.
      \item Dogovor, da se bo prostor najel čim prej in se posledično reči lahko začnejo odvijati hitreje. Tudi tu čakamo na naročilnice s strani ŠO FMF.
  \end{itemize}
  
\item
  Poročilo člana ŠS FMF o pogovoru z namestnico za študijske zadeve na matematiki o določeni pritožbi zoper dotičnega kandidata in njeno vprašanje o lanskem izdanem študentskem mnenju zanj.
  
  Sporočena je pritožba o manjkajoči tabli v eni izmed kletk v 4.~nadstropju.
  
  Na 1.~seji je bila omenjena Moderna fizika. Klemen poroča o dopisu, ki ga je poslal dekanu. Dekan dopis s pobudo še proučuje.
  
\end{ad}

% spremeni footer
\makeatletter \global\let\@enddocumenthook\@empty \makeatother
\AtEndDocument{
  Zapisala Neža Dimec, pregledal Klemen Ducman

  \begin{flushright}
    Predsednik ŠS FMF \\
    Klemen Ducman
\end{flushright}}

\end{document}
