\documentclass{seja}

\newcommand{\SejaNum}{5}
\newcommand{\Title}{Zapisnik \SejaNum.~redne seje Študentskega sveta FMF}
\newcommand{\DateSeja}{26.~2.~2017}
\newcommand{\DateZapisnik}{7.~3.~2017}

\begin{document}

Prisotni člani in članice ŠS FMF: Klemen Ducman, Rok Gregorič, Ana Smerdu, Miha Šifrer, Ines Meršak, Saša Marolt, Jure Slak, Mojca Žilavec

Opravičeno odsotni: Miha Srdinšek, Marion van Midden

Neopravičeno odsotni: Gaja Guerrini, Ana Kregar

\begin{red*}
\item Sprejem zapisnikov prejšnjih sej
\item Večeri družabnih iger in javna tribuna
\item Pi dan (organzacija, predstava, komisija, prijave, nagrade)
\item Organizacija in čiščenje študentske sobe ter odpiranje omar
\item Poročila predstavnikov v organih
\item Debata in sprejemanje študentskih mnenj
\item Seznanitev s situacijo glede MF in ostale težave s profesorji na fiziki
\item Poročilo o projektih (izrek dneva, hash code, matematični hoodie)
\item Članstvo Gaje Guerrini
\end{red*}

\begin{ad}
\item
\begin{sklep*}
  ŠS FMF potrjuje zapisnik 4.~seje.
\end{sklep*}

\item
Predstavnica ŠOU Ajda nam predstavi, da bo v četrtek, 2.\ 3. prvi večer družabnih iger v Mafiji. Potekal bo od 18.00 - 21.00. Vabljeni smo, da pridemo na večer družabnih iger. Če želimo pa lahko pregledamo tudi navodila iger, da bo igranje potekalo lažje.
V ponedeljek, 6.\ 3. bo potekala javna tribuna. Pomembno je, da se je udeležimo, saj imamo s strani ŠOU financiranje majice za mafijski piknik.
Poraja se vprašanje, da moramo ugotoviti, kako bomo izdajali račune za mafijski piknik. Sklep je, da se Jure pozanima o tej stvari.

\item
Pi dan bo ptekal 14.\ 3. ob 13h. Organizacija poteka s strani ŠOU, glavni organizator pa je Grega. Ta poskrbi za nagrade in diplome (ideja kuponov za pijačo v mafiji po 3,14 evrov). Prav tako se ustvari dokument s pravili, vabilo in dogodek na facebooku. Prijave bodo potekale vnaprej preko obrazca, v prijavnici pa zahtevamo ime in priimek, šolo, priblžno število povedanih decimalk.
Hrane in pijače ni potrebno kupovati, saj je je veliko v študentski sobi. Odločimo se za nakup 10 pic.
Okoli 17h bo potekala prestava Nika, imenovana "Naj gre vse v pi", katere plačilo delno krije FMF (upravni odbor).
Pred tekmovanjem Grega pošlje spisek z vsem tekmovalci, spisek oseb v komisiji in pripravi potrebno gradivo.

\item
Po debati ugotovimo, da Ajda pozna nosilca kljča omare v študentski sobi, zato se jo zaprosi, da ta ključ prinese in se pogleda, kaj je v omari. Računalnike, ki nisi od ŠOU lahko nesemo nazaj v RC.
Dogovorimo se, da se po pi dnevu 2 člana ŠS IN 2 člana ŠOU dobijo v študentski sobi, pregledajo stvari in kar ni uporabno očistijo.
Jure po pi dnevu skliče sestanek za pospravljanje študentske sobe.

\item
Poročilo z ZPS fizike je bilo poslano po elektronski pošti in ni vsebovalo večjih posebnosti.

Senat (Ines): letos potekajo volitve za dekana, hkrati pa je pomembna informacija da vsi študentje volimo rektorja.

ŠSUL (Klemen): Za izdajanje mnenj, moramo vedno pridobiti vsa tekstkovna mnenja.

\textit{Sejo zapusti Ana Smurdu. Na seji se pridruži Rok Gregorič.}

ZPS (Rok): sprememba pravilnika za prešernove nagrade. Nov pravilnik Rok pošlje Juretu v pogled.

\item
Svet je seznanjen z mnenji.

\begin{sklep*}
Svet sprejme mnenje o Dejanu Veluščku.
\end{sklep*}
\begin{sklep*}
Svet sprejme mnenje o Nedeljki Žagar.
\end{sklep*}
\begin{sklep*}
Svet sprejme mnenje o Danielu Svenšku.
\end{sklep*}
\begin{sklep*}
Svet sprejme mnenje o Sašu Strletu.
\end{sklep*}
\begin{sklep*}
Svet sprejme mnenje o Matiji Vidmarju. Jure se dogovori za pogovor.
\end{sklep*}
\begin{sklep*}
Svet sprejme mnenje o Luki Boc Thalerju.
\end{sklep*}

\item
Za moderno fiziko se napiše dopis, kjer se uporabi pristop, da je tak pristop nepravilen, saj pri ostalih predmetih tega ni več. Klemen Ducman do začetka marca spiše dopis in ga pošlje v pregled ŠS. Nato se dopis pošlje prof.\ Golobu.

\item
Na matematiki se redno izvaja projekt izrek dneva. Hkrati smo izvedli projekt Hash Code, ki je bil uspešen z udeležbo 60ih tekmovalcev.
Matematični hoodie je v nastajanju. Cena bo znašala 30 evrov.

\item
Gaja Guerrini je neodzivna na delo ŠS, prav tako pa se ne odzove na osebne maile s strani vodstva ŠS.

\begin{sklep*}
ŠS soglasno sprejme, da Gaja Guerrini ni več članica ŠS FMF.
\end{sklep*}

Zaradi tega moramo nadomestiti vse funkcije, za katere je bila zadolžena Gaja Guerrini.

\begin{sklep*}
Miha Šifrer postane predstavnik v komisija za etična vprašanja.
Namestnica v komisiji za kakovost postane Saša Marlot.
Jure Slak bo spremembo sporočil vodstvu fakultete.
\end{sklep*}

\end{ad}

\end{document}
