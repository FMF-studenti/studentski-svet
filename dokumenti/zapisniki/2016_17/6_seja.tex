\documentclass{seja}

\newcommand{\SejaNum}{6}
\newcommand{\Title}{Zapisnik \SejaNum.~redne seje Študentskega sveta FMF}
\newcommand{\DateSeja}{27.~3.~2017}
\newcommand{\DateZapisnik}{18.~4.~2017}

\begin{document}

Prisotni člani in članice ŠS FMF: Klemen Ducman, Ana Kregar, Ines Meršak, Jure Slak, Ana Smerdu, Miha Srdinšek, Miha Šifrer, Marion van Midden

Opravičeno odsotni: Rok Gregorič, Saša Marolt, Mojca Žilavec

\begin{red*}
\item Sprejem zapisnika prejšnje seje
\item Poročila predstavnikov v organih
\item Debata in sprejemanje študentskih mnenj
\item Določitev predstavnikov študentov v Akademskem zboru in predstavnikov za volitve dekana in rektorja
\item Dopolnitev poslovnika glede volitev člana v upravni odbor in konkretna izvolitev
\item Mafijski piknik
\item Nova knjižnica
\item Poročilo o projektih (pi dan)
\item Uvedba in formalizacija ``predstavnikov letnikov'' za lažjo komunikacijo
\end{red*}

\begin{ad}
\item
\begin{sklep*}
  ŠS FMF potrjuje zapisnik 5.~seje.
\end{sklep*}

\item Senat (Klemen, Ines): Po novem naj bi morali vsi organi (tudi študentski svet) članice univerze na svoje seje vabiti tudi predstavnico sindikata. Volitve rektorja bodo potekale 24.~maja, na fakulteti bodo 4 volišča, za vsako volišče pa je potreben tudi en predstavnikov študentov.

ZPS (Miha Srdinšek): Na ZPS fizike je prišla pobuda, da se pred stavbo Oddelka za fizike grmovje nadomesti s travo, na kateri bi lahko študentje sedeli med odmori. ŠS poda še par drugih možnosti, ki bodo morda lažje izvedene: na kamnito klopco med obema stavbama bi se lahko namestilo les, da bi na njej brez težav lahko sedeli tudi pozimi. Lahko bi odprli teraso na vrhu Ma$\varphi$je ali pa sedeli na travi pred meteorološko hiško, če bi bila ta seveda redno košena. Jure predloge pošlje vodstvu fakultete.

\item
Svet je seznanjen z mnenji.

\begin{sklep*}
Svet sprejme mnenje o Mitji Krnelu.
\end{sklep*}
\begin{sklep*}
Svet sprejme mnenje o Marku Medenjaku.
\end{sklep*}

\item
\begin{sklep*}
ŠS potrdi svoje člane za predstavnike v Akademskem zboru, predstavniki so torej:
Klemen Ducman, Rok Gregorič, Ana Kregar, Saša Marolt, Ines Meršak, Jure Slak, Ana Smerdu, Miha Srdinšek, Miha Šifrer, Marion Antonia van Midden in Mojca Žilavec.
\end{sklep*}

\begin{sklep*}
ŠS potrdi Sašo Marolt za članico volilnega odbora.

ŠS potrdi naslednje predstavnike študentov na voliščih:
\begin{itemize}
    \item \textit{Sejna soba dekanata, Jadranska 19:} Ana Kregar, namestnik Klemen Ducman
    \item \textit{Študentski klub, Jadranska 19:} Klemen Ducman, namestnica Marion van Midden
    \item \textit{Sejna soba, Jadranska 21:} Ana Smerdu, namestnica Saša Marolt
    \item \textit{Skupna soba, Jadranska 21:} Ines Meršak, namestnik Jure Slak
\end{itemize}
\end{sklep*}

\item
Klemen v poslovnik doda amandma o predstavniku študentov v upravnem odboru, v kateremu se določi, da je mandat predstavnika 1 leto in da za to pozicijo ne more kandidirati študent prvega letnika.

\begin{sklep*}
Klemen Ducman postane predstavnik v upravnem odboru, Miha Šifrer pa njegov namestnik.
\end{sklep*}

\item
Dogovorimo se za možne datume Mafijskega piknika, po preferenci so to 17.~maj, 18.~maj, 10.~maj in 11.~maj. Marion pokliče v Mostec in preveri, kateri termini so še na voljo in kakšna bo cena najema lokacije. Debatira se tudi o možnosti sponzoriranja s strani Ma$\varphi$je, nad čimer pa vodstvo lokala ni navdušeno. Po drugi strani ŠS skrbi prodaja vstopnic; ena možnost je, da jih prodaja kar Ma$\varphi$ja, nad čimer pa člani niso močno navdušeni, saj bi to pomenilo podražitev. Druga možnost je ustanovitev društva, prek katerega bi lahko ŠS prodajal vstopnice --  sklep je, da se Ana in Klemen pozanimata, kaj vse je za to potrebno.

\item
Svet se seznani z načrti za novo knjižnico. Jure pošlje pripombe na načrt vodstvu: čitalnica naj bo zvočno izolirana, ima velike mize, ki niso ločene, na mizah naj bodo (če je to mogoče) svetilke za boljšo vidljivost. Poskrbeti je treba za veliko električnih vtičnic tako v čitalnici kot v skupnem prostoru, po možnosti naj bo ena vtičnica na en delovni prostor.

\item
Pi dan je bil uspešno izveden in dobil več medijske pozornosti kot do zdaj, kljub temu da rekorda ni bilo (najboljši rezultat je bil manj kot 1000 decimalk). Po recitiranju je aktualni slovenski rekorder Nik Škrlec izvedel predstavo v 2.05 na Oddelku za matematiko, ki je bila deležna visoke udeležbe.

Ana Smerdu pripomni, da bo naslednji teden feministični, zato bodo potekali različni dogodki, kot npr.~impro predstava in filmski večer.

\item
Klemen v poslovnik doda amandma o predstavnikih letnikov. Naša želja je, da fakulteta zbere predstavnike letnikov za vsak program posebej, potem pa jih sporoči ŠS, ki hrani imena predstavnikov letnikov. Klemen in Ines se po naslednji seji senata v zvezi s tem dogovorita z dekanom in obema predstojnikoma.

\item
Klemen in Ines natisneta potrebne plakate za oglaševanje programerskega tekmovanja UPM, natečaja za mafijsko majico in prostovoljstva.

Jure se pri vodstvu Oddelka za matematiko pozanima, ali bi lahko na vrh internih anket napisali, da je nujno izpolniti tudi anketo o delu asistentov in profesorjev tudi v VISu, saj s pomočjo rezultatov teh anket ŠS izdaja mnenja. Steče tudi debata, kako študente fizike pripraviti do tega, da rešujejo ankete na VISu -- sklep je, da Jure nereševanje omeni predstojniku za študijske zadeve.

Svet se seznani, da je Vlado Malačič zaprosil za mnenje za področje meteorologije.

Klemen predlaga, da bi člani ŠS FMF potrebovali uraden template za podpis v elektronsko pošto. Sklep je, da da Jure template v poslovnik, Klemen pa napiše, kakšne so kazni za zlorabo podpisa.

\end{ad}

% spremeni footer
\makeatletter \global\let\@enddocumenthook\@empty \makeatother
\AtEndDocument{
Zapisala Ines Meršak, pregledal Jure Slak

\begin{flushright}
  Predsednik ŠS FMF \\
  Jure Slak
\end{flushright}}


\end{document}
