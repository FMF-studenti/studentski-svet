\documentclass{seja}

\newcommand{\SejaNum}{8}
\newcommand{\Title}{Zapisnik \SejaNum.~redne seje Študentskega sveta FMF}
\newcommand{\DateSeja}{8.~6.~2017}
\newcommand{\DateZapisnik}{20.~6.~2017}

\begin{document}

Prisotni člani in članice ŠS FMF: Klemen Ducman, Rok Gregorič, Ana Kregar, Saša Marolt, Ines Meršak, Jure Slak, Ana Smerdu, Miha Šifrer, Marion van Midden, Mojca Žilavec

Opravičeno odsotni: Miha Srdinšek

\begin{red*}
\item Sprejem zapisnika prejšnje seje
\item Poročila predstavnikov v organih
\item Debata in sprejemanje študentskih mnenj
\item Poročilo mafijskega piknika in naročanje dodatnih majic
\end{red*}

\begin{ad}
\item
\begin{sklep*}
  ŠS FMF potrjuje zapisnik 7.~seje.
\end{sklep*}

\item
Senat (Klemen, Ines): Sledila je standardna seja senata s poročilom dekana.

ZPS (Miha Srdinšek): Na fiziki bodo prenavljali stranišča in dvigala, zato bodo pred fakulteto kemična stranišča. Nov predstojnik fizike bo s študijskim letom 2017/18 D.\ Arčon. Prav tako bomo z novim letom uvedli mesečne sestanke skrbnikov programov. Med 21.-25.8. se bo odvila matematično-fiziična šola za dijake 1. in 2. letnika srednjih šol.

ZPS (Rok Gregorič): Nov predstojnik matematike bo s študijskim letom 2017/18 P.\ Potočnik. Odgovoren za finance in računovodstvo bo T.\ Košir. Ponovno se je na dobatiralo o prepisovanju med študenti na OM in med vodstvom je bilo
nekaj dvomov ali je sploh potrebno se s tem ukvarjati.
\begin{sklep*}
Jure Slak opozori novo vodstvo na obstoj problematike z vidika študentov preko elektronske pošte.
\end{sklep*}
Na programu IŠRM se s študijskim letom 2017/18 omeji vpis za študente izven Evropske unije na 4 mesta.

SŠUL (Klemen Ducman, Ana Kregar): Ustanovili bodo disciplinsko komisijo, ki bo lahko nase prevzela
problematiko prepisovanja.

\item
Klemen do ponedeljka, 12.\ 6.\ 2017 napiše dopis za dekana v zvezi s problematiko diplomskih in magistrskih nalog.

\item
Svet je seznanjen z mnenji.

\begin{sklep*}
Svet sprejme mnenje o Ambrožu Kregarju.
\end{sklep*}
\begin{sklep*}
Svet sprejme mnenje o Blažu Jelencu Longerju.
\end{sklep*}
\begin{sklep*}
Svet sprejme mnenje o Emilu Polajnarju.
\end{sklep*}
\begin{sklep*}
Svet sprejme mnenje o Janu Grošlju.
\end{sklep*}
\begin{sklep*}
Svet sprejme mnenje o Katarini Kosovelj.
\end{sklep*}
\begin{sklep*}
Svet sprejme mnenje o Mateju Preglju.
\end{sklep*}
\begin{sklep*}
Svet sprejme mnenje o Mihi Mihoviloviču.
\end{sklep*}
\begin{sklep*}
Svet sprejme mnenje o Milošu Borovšaku.
\end{sklep*}
\begin{sklep*}
Svet sprejme mnenje o Niki Novak.
\end{sklep*}
\begin{sklep*}
Svet sprejme mnenje o Pavletu Saksidi.
\end{sklep*}
\begin{sklep*}
Svet sprejme mnenje o Roku Žitku.
\end{sklep*}

\item
Jure predstavi poročilo mafijskega piknika. Razvila se je debata, da lahko mafijski piknik
prijavimo kot projekt na UL in dobimo določeno vsoto denarja za sofinanciranje.
Po izkušnjah organizatorjev piknika se naslednje leto v študentskem svetu ustanovi
organizacijski odbor za mafijski piknik, ki bo prevzel delo in organizacijo dogodka.

Majice za mafijski piknik so pošle, zato je potrebno dodatno naročilo.
Prav tako bomo svoje številk majic oddale čistilke. Ko majice prejmemo
jih je potrebno razdeliti vsem, ki jih na mafijskem pikniku niso prejeli.
Če bo kdo želel dodatno majico jih bomo zaračunavali po 5 eur.

\begin{sklep*}
Ana Kregar prijavi projekt Mafijski piknik na UL za naslednje leto.
\end{sklep*}

\item
S problematiko ustanavljanja društva se ukvarja ŠS za leto 2017/18.

Usposabljanja za LGTB se bosta udeležila Saša M.\ in Miha S.

\end{ad}

\end{document}
