\documentclass{seja}

\newcommand{\SejaNum}{10}
\newcommand{\Title}{Zapisnik \SejaNum.~redne seje Študentskega sveta FMF}
\newcommand{\DateSeja}{25. 10. 2017}
\newcommand{\DateZapisnik}{25. 10. 2017}

\begin{document}

Prisotni člani in članice ŠS FMF: Klemen Ducman, Ana Kregar, Ines Meršak, Jure
Slak, Miha Srdinšek, Miha Šifrer, Marion van Midden

Opravičeno odsotni: Saša Marolt, Ana Smerdu, Mojca Žilavec

\begin{red*}
\item Poročila predstavnikov v organih
\item Sprejem študentskih mnenj
\item Predog sprememb poslovnika ŠS FMF
\item Volitve v ŠS FMF
\item Kolokviji pri Moderni fiziki
\item Predlog Red Bull dogodka z električnimi avtomobili
\item Dogodek ``Delo raziskovalca fizike''
\end{red*}

\begin{ad}
\item Senat (Ana K., Ines, Klemen): trajanje doktorskega študija na UL se bo
najverjetneje podaljšalo iz 3 let na 4 leta. Na izredni oktobrski seji senata
so bile obravnavane pritožbe na zavrnjene prošnje za izredno podaljšanje
statusa. Z novim študijskim letom začnejo veljati bolj ostri pogoji za
uveljanje pravic ob študiju (zdravstveno zavarovanje, študentska prehrana
\ldots) -- študentu pripada $n+1$ let pravic, kjer je $n$ trajanje (v letih)
njegovega študijskega programa.

\item
Svet je seznanjen z mnenji.

\begin{sklep*}
Svet sprejme mnenje o Angeli Kochoski.
\end{sklep*}
\begin{sklep*}
Svet sprejme mnenje o Emilu Žagarju.
\end{sklep*}
\begin{sklep*}
Svet sprejme mnenje o Gašperju Jakliču.
\end{sklep*}
\begin{sklep*}
Svet sprejme mnenje o Gregorju Kladniku.
\end{sklep*}
\begin{sklep*}
Svet sprejme mnenje o Gregorju Posnjaku.
\end{sklep*}
\begin{sklep*}
Svet sprejme mnenje o Jaku Muru.
\end{sklep*}
\begin{sklep*}
Svet sprejme mnenje o Juretu Aplincu.
\end{sklep*}
\begin{sklep*}
Svet sprejme mnenje o Klemnu Bučarju.
\end{sklep*}
\begin{sklep*}
Svet sprejme mnenje o Matiji Milaniču.
\end{sklep*}
\begin{sklep*}
Svet sprejme mnenje o Mihu Mihoviloviču.
\end{sklep*}
\begin{sklep*}
Svet sprejme mnenje o Roku Hrenu.
\end{sklep*}

\item Predlogi sprememb poslovnika:
\begin{description}
  \item[8.~člen (sprememba)] ``Vsak član Sveta lahko uporablja tudi svojo
  digitalno vizitko, ki vsebuje logotip, članstva v organih in komisijah, ter
  kontaktne informacije. Vsak član je dolžan zagotoviti pravilnost podatkov na
  vizitki. Navajanje lažnih podatkov z namenom, da bi neupravičeno okoristil sebe
  ali koga drugega šteje za hujšo kršitev in se obravnava po Pravilniku o
  disciplinski UL. Študent, ki ni član Sveta in zlorabi vizitko prav tako
  odgovarja za enako kršitev.''
  \item[12.~člen (dodan)] ``Kandidat ob vložitvi kandidature ne sme biti pravnomočno
  obsojen v disciplinskem postopku. Prav tako za člana Sveta ne more biti izvoljen član
  katerekoli Študentske organizacije visokošolskega zavoda, član Študentskega
  zbora ali član Študentske organizacije katerekoli univerze ali Študentske
  organizacije Slovenije. Prav tako za člana Sveta ne more biti izvoljen študent,
  ki opravlja kakšno izvršno funkcijo na katerikoli ravni (morda študentske) politične
  stranke ali podmladku politične stranke ali se aktivno politično udejstvuje
  (npr.~volilne kampanije, referendumske kampanije, ipd.).
  Prav tako ne sme biti član izvršnih organov verskih skupnosti ali ustanov.'' \\
  (\emph{Opomba}: zaradi novega člena se zamaknejo številke vseh kasnejših členov.)
  \item[14.~člen (sprememba, novi 15.)] Naslov člena se spremeni iz besedila
  ``(dnevni red ustanovne seje)'' na ``(dnevni red konstitutivne seje)''.
  \item[18.~člen (sprememba, novi 19.)] V odstavek ``Če se član dvakrat
  neupravičeno ne udeleži seje Sveta, DT ali organa Univerze, ali stori kakšno
  drugo hujšo kršitev, ga lahko na naslednji seji Svet z večino vseh članov
  razreši. Član za kršitev odgovarja tudi po Disciplinskem pravilniku ŠS UL.'' se
  doda ``ali kakšno drugo hujšo kršitev [...] Član za kršitev odgovarja tudi po
  Disciplinskem pravilniku ŠS UL.''
  \item[19.~člen (sprememba, novi 20.)] Odstranjeni so bili sklici na člene
  Poslovnika ŠS UL, saj se ta spreminja.
  \item[20.~člen (sprememba, novi 21.)] Predzadnji odstavek 20.~člena se zaradi
  novega 12.~člena okrajša na: ``Kandidat ima pravico pred začetkom volitev
  obrazložiti predlog kandidature oz.~kandidaturo. Za predsednika sveta ne more
  biti izvoljen član katerekoli, niti študentske, politične stranke ali podmladka
  politične stranke.''
\end{description}

\begin{sklep*}
Svet sprejme zgornje dopolnitve k poslovniku.
\end{sklep*}
Dvotretjinska večina je dosežena.

\item Svet se z vodstvom FMF v skladu s poslovnikom dogovori za volitve
naslednjega študentska sveta.

\item Svet se seznani s situacijo pri predmetu Moderna fizika 1: kljub
lanskoletnemu dogovoru z (bivšim) vodstvom fakultete in izvajalcem predmeta, da
popravnih kolokvijev ne bo več, jih izvajalec ponovno namerava uporabiti kot
način ocenjevanja znanja.

\begin{sklep*}
Klemen napiše dopis novemu dekanu, v katerem ga seznani s situacijo in poprosi
za nadaljnjo reševanje tega problema.
\end{sklep*}

\item Svet po kratki debati v zvezi z Red Bull dogodkom sprejme spodnji sklep.
\begin{sklep*}
Pomoč pri organizaciji Red Bull dogodka ni v pristojnosti ŠS FMF, saj to ni
študijska zadeva. Jure organizatorjem dogodka zato posreduje email naslov člana
Študentske organizacije FMF.
\end{sklep*}

\item Ana K.~poroča, da je bil dogodek uspešen in dobro sprejet. Na dogodku je
na kratko tudi predstavila delo ŠS FMF in podarila, da take dogodke podpiramo.
Matematični del sveta si želi izvedbo podobnega dogodka tudi na matematični
strani.

\item
\begin{sklep*}
Svet sprejme zapisnik 10.~seje.
\end{sklep*}

\begin{sklep*}
  Svet izda članom Sveta potrdila o članstvu.
\end{sklep*}
\end{ad}

% spremeni footer
\makeatletter \global\let\@enddocumenthook\@empty \makeatother
\AtEndDocument{
Zapisala Ines Meršak, pregledal Jure Slak
\begin{flushright}
  Predsednik ŠS FMF \\
  Jure Slak
\end{flushright}}

\end{document}
