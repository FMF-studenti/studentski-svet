\documentclass{seja}

\newcommand{\SejaNum}{7}
\newcommand{\Title}{Zapisnik \SejaNum.~redne seje Študentskega sveta FMF}
\newcommand{\DateSeja}{19.~4.~2017}
\newcommand{\DateZapisnik}{20.~4.~2017}

\begin{document}

Prisotni člani in članice ŠS FMF: Klemen Ducman, Rok Gregorič, Ana Kregar, Saša Marolt, Ines Meršak, Jure Slak, Ana Smerdu, Miha Srdinšek, Miha Šifrer

Opravičeno odsotni: Marion van Midden, Mojca Žilavec

\begin{red*}
\item Sprejem zapisnika prejšnje seje
\item Poročila predstavnikov v organih
\item Debata in sprejemanje študentskih mnenj
\item Amandmaji v poslovniku (upravni odbor, logotipi, spletna stran, podpisi)
\item Uvedba predstavnikov letnikov (poročilo + dodatek v poslovniku)
\item RC
\item Mafijski piknik
\end{red*}

\begin{ad}
\item
\begin{sklep*}
  ŠS FMF potrjuje zapisnik 6.~seje.
\end{sklep*}

\item
Senat (Klemen, Ines): V senatu UL se še niso uspeli dogovoriti za končno verzijo študijskega koledarja za naslednje študijsko leto, Anton Ramšak je bil izvoljen za dekana, na Akademskem zboru so bili izvoljeni člani senata FMF za prihodnji dve leti.

ZPS (Miha Srdinšek): Uvedba novih nagrad Roberta Blinca za področje fizike predvsem za mlajše doktorje fizike, veliko so se prerekali, kdaj bodo zaostreni pogoji za mnogokratno ponovno izvolitev v asistenta.

\item
Jure svet seznani o dogovoru v zvezi z ukrepi za Matijo Vidmarja.

\item
Jure in Klemen svet seznanita z amandmaji, ki sta jih dodala v poslovnik: o izvolitvi in mandatu predstavnika študentov v upravnem odboru, o logotipu študentskega sveta in o templateu podpisa za v e-mail, prav tako pa so bile dodane kazni za zlorabo podpisa.
\begin{sklep*}
    ŠS FMF potrjuje amandmaje poslovnika.
\end{sklep*}

\item
Klemen in Ines svet seznanita z dogovorom, ki je bil dosežen po seji senata FMF, v zvezi s predstavniki letnika. Klemen doda člen o predstavnikih letnika v poslovnik: namen predstavnikov letnikov je lažja komunikacija z vsakim letnikom vsakega programa posebej, imena predstavnikov letnika in njihove e-mail naslove ŠS FMF objavi na svoji spletni strani \url{svet.fmf.si}.

\item
Steče debata o tem, kaj vse bi se lahko napisalo v pritožbo nad računalniškim centrom, ki jo bo spisal Rok. ŠS FMF meni, da bi RC lahko izboljšal naslednje stvari:
\begin{itemize}
    \item \textit{hitrejša odzivnost računalnikov ob prijavi:} Ob prijavi mora študent čakati 5 min, da lahko pride do gradiv na spletni učilnici (to je dejanski čas, izmeril ga je Jure Slak). Posledično v enem odmoru (15 min) študent pogosto ne uspe natisniti gradiv, ki jih potrebuje za naslednjo uro ali pa za preverjanje znanja.
    \item \textit{odprava problema s prevelikim profilom:} Praktično vsak študent, ki je na fakulteti več kot pol leta in je aktivno uporabljal računalnik (torej med vajami v računalniški učilnici) ima problem s prevelikim profilom, kar pogosto moti pedagoški proces, saj se problemi s tem pojavljajo predvsem med delom v IDEjih, ki pa so potrebni pri večini računalniških predmetov.
    \item \textit{ponovna postavitev računalnikov pred 2.05:} Pred 2.05 sta bila včasih dva računalnika, ki sta bila koristna predvsem, ko je študent imel predavanja v drugem nadstropju in je le želel nekaj na hitro preveriti (npr.~rezultate kolokvija ali kviza).
    \item \textit{popravilo nekaterih računalnikov v 3.08 in 3.09:} Vsaj en računalnik (ali pa morda zaslon) je prenehal delovati v učilnici 3.09, poleg tega je v učilnici 3.08 vsaj en računalnik, katerega zaslon vse barve prikazuje kot modro-rumene.
    \item \textit{spletna stran FMF naj dela tudi, če spredaj ne napišeš www:} \url{fmf.uni-lj.si} trenutno ne deluje, kar je za študente moteče.
    \item \textit{spletna stran FMF naj privzeto uporablja HTTPS protokol:} Trenutno spletna stran privzeto uporablja HTTP protokol, ki pa ni tako varen.
    \item \textit{popravki novih urnikov:} Mobilna verzija na telefonu ni najbolj pregledna, saj skrči nekatere predmete do mere, ko niso več berljivi; prav tako si študentje želimo poenotenje kratic predmetov s spletno učilnico.
    \item \textit{plačilo domene fmf.si:} Domena \url{fmf.si} bo počasi potekla, zato je potrebno plačati podaljšanje domene za naslednje leto.
\end{itemize}

\item
Svet se seznani z lokacijo (Mostec) in datumom (17.~maj) Mafijskega piknika. Lokacija stane 400 evrov, denar za lokacijo bo prišel s strani faksa prek naročilnice za 1500 evrov. Preostalih 1100 evrov lahko porabimo za nealkoholne potrebščine za piknik. Če se ustanovitev društva ne bo realizirala dovolj zgodaj, bomo karte prodajali sami.

Letos se odločimo za nakup manjše količine ($40 l + 40 l$), a zato bolj kakovostne vrste vina. Kot lani bomo nakupili 40 platojev radlerjev, po večini okusa grenivke, in 90 platojev piva (45 temno + 45 svetlo). Ideja za točenje nefiltriranega Uniona naleti na veliko entuziazma, Ana K.~se javi, da bo poizvedela o ponudbi. Jure za meso zadolži / povpraša Tadeja, ki je za to skrbel tudi v preteklih letih. Jure prav tako povpraša Marion, kako je bilo lani poskrbljeno za WC -- tudi letos imamo namen najeti dve kabini.

Predebatirajo se tudi možne rešitve za zmanjšanje vrste za hrano, pade ideja o številkah za hrano in števcu za odbojko, s pomočjo katere bi lahko sledili, katere številke so na vrsti za hrano. Išče se tudi mreža za odbojko, katero se je lani napeljalo med drevesa.

Miha S.~poskrbi, da bo Iskra dala natisniti majice za piknik, Jure in Ines na pikniku poskrbita za športne dogodke, Ana S.~se javi, da bo šla v trgovino.

\item
Klemen in Ana K.~poročata o napredku v zvezi z ustanovitvijo društva. Potrebno je napisati statut društva in zapisnik ustanovne seje; strošek registracije društva bo znašal 32 evrov, strošek spremembe sedeža društva pa 9 evrov (sedež društva mora potrditi senat UL). Za računovodsko društvo bi morali plačati med 50 in 100 evrov vsako leto. Svet sprejme, da se splača nadaljevati in ustanoviti društvo. Klemen poskrbi, da bo društvo čim prej registrirano.

\end{ad}

% spremeni footer
\makeatletter \global\let\@enddocumenthook\@empty \makeatother
\AtEndDocument{
Zapisala Ines Meršak, pregledal Jure Slak

\begin{flushright}
  Predsednik ŠS FMF \\
  Jure Slak
\end{flushright}}


\end{document}
