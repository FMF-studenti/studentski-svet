\documentclass{seja}

\newcommand{\SejaNum}{3}
\newcommand{\Title}{Zapisnik \SejaNum.~redne seje Študentskega sveta FMF}
\newcommand{\DateSeja}{20.~12.~2016}
\newcommand{\DateZapisnik}{18.~1.~2017}

\begin{document}

Prisotni člani in članice ŠS FMF: Klemen Ducman, Ana Kregar, Ines Meršak, Rok Gregorič, Miha Podkrajšek, Ana Smerdu, Miha Srdinšek, Miha Šifrer, Marion van Midden in Mojca Žilavec.

Opravičeno odsotni: Jure Slak, Saša Marolt

Neopravičeno odsotni: Gaja Guerrini

\begin{red*}
\item Sprejem zapisnikov prejšnjih sej
\item Sprejem študentskih mnenj
\item Poročila predstavnikov v ZPS-jih in senatu
\item Debata o mnenju o dr.~Veluščku
\item Odsotnost Mihe Podkrajška pri vseh aktivnostih in sejah sveta
\item Pritožba glede ocenjevanja pri Moderni fiziki
\item Pritožba glede naloge pri Verjetnosti 1
\end{red*}

\begin{ad}
\item
\begin{sklep*}
  ŠS FMF potrjuje zapisnik 2.~seje.
\end{sklep*}

\item
Svet je seznanjen z mnenji.

\begin{sklep*} ŠS FMF sprejme mnenje o Sergeju Faletiču. \end{sklep*}
\begin{sklep*} ŠS FMF sprejme mnenje o Primožu Ziherlu. \end{sklep*}
\begin{sklep*} ŠS FMF sprejme mnenje o Sergeju Faletiču. \end{sklep*}

Ostala mnenja sprejmemo na naslednji seji.

\item
Poročilo senata in ZPS-ja se nahajata v priloženem dokumentu.
Ana S., Marion, Ana K., Ines in Klemen se dogovorijo za sestanek z dekanom glede
predloga novega statuta UL. Sestanek skooridinira Marion.

\textit Sejo zapusti Rok Gregorič.

\item
V dodatnem dokumentu (poslan s strani Ines) je razlaga o dr. Veluščku in
pogovori v senatu, dne 14.~12.~2016.  Svet na tej točki sklene, da se
posamezniki dobijo na sestanku z dr. Veluščkom, kjer ga seznanimo s študentskimi
mnenji. Jure organizira sestanek po novem letu, na katerem je prisotna tudi
Marion.

\item
Debata o Mihi Podkrajšku ni potrebna, saj je prisoten na seji.

\item
Problem predmeta Moderna fizika je v tem, da se pri predmetu ne izvajajo izpiti
kot ocenjevanje znanja.  Svet se odloči, da predlaga to točko kot dnevni red na
naslednjem ZPS-ju. Hkrati pa se svet odloči, da Klemen pripravi dopis do
24.~12.~2016, v katerem pojasnjuje o nastali situaciji. Končno verzijo dopisa
Jure pošlje dekanu in profesorju Golobu.

\item
S strani študentke je prišla pritožba o neprimerni nalogi pri Verjetnosti 1
(smer matematika), ki naj bi poudarjala seksizem (neprimeren odnos do žensk).
Svet se odloči, da se Jure po novoletnih praznikih dobi na sestanku z
asist.~Raičem in se pogovorita o primernosti naloge.  Naloga pa je bila s strani
profesorja in asistenta pri Verjetnosti 1 že odstranjena z učnega načrta.

\item
Vsak izmed članov sveta do 23.~12.~2016 napiše samoevalvacijsko poročilo o tem,
kaj fakulteta lahko izboljša in kaj bo svet za to naredil. Govorilo se je že o
Mafijskem pikniku in $\pi$-dnevu.

Dogovorili smo se še zadnje malenkosti glede novoletne stojnice.

Naslednja seja bo po novoletnih praznikih, v začetku januarja 2017.
\end{ad}

\end{document}
