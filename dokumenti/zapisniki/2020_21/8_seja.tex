\documentclass{seja}

\newcommand{\SejaNum}{8}
\newcommand{\Title}{Zapisnik \SejaNum.~redne seje Študentskega sveta FMF}
\newcommand{\DateSeja}{27.~9.~2021}
\newcommand{\DateZapisnik}{3.~10.~2021}

\begin{document}

Prisotni člani in članice ŠS FMF UL:
Matej Janežič,
Matevž Kopač,
Žan Mikola,
Aljoša Rebolj,
Katarina Šipec,
Matic Tonin,
Jakob Zmrzlikar,
Nastja Zupančič,
Gašper Žajdela.

Opravičeno odsotni člani in članice ŠS FMF UL:
Klara Širca,
Tjaša Vagaja,
Nina Velkavrh.

\begin{red}
    \item
    Potek mandatov članov ŠS FMF
	\item
	Sprejem zapisnikov prejšnjih sej 
	\item
	Poročilo predsedstva
	\item
	Poročila predstavnikov v organih in komisijah 
    \item
    Sprejem študentskih mnenj
    \item
    Imenovanje novega predsedstva 
    \item
    Spremembe poslovnika ŠS FMF
    \item
    Razno
\end{red}



\begin{ad}
    \item 
    ŠS FMF se seznani, da Klemenu Ducmanu zaradi izgube statusa študenta poteče mandat v ŠS FMF.
    
    \item
    \begin{sklep*}
        ŠS FMF sprejme zapisnike prejšnjih sej.
    \end{sklep*}
    
    \item
    Predsednik ŠS FMF pove, da se je v sredo, 22. 9. 2021, odvijala pedagoška konferenca FMF, na kateri so bili prisotni tudi predstavniki ŠS FMF (Matej Janežič, Katarina Šipec in Nastja Zupančič). Glavna tema konference je bila izvajanje pedagoških  procesov v novem študijskem letu -- izvajanje pogoja PCT, dodatne možnosti spremljanja pouka na daljavo (ki naj bi sedaj potekal izključno v živo) in pa uvedba svetovalnice za študente v duševni stiski. Za mnenje so bili vprašani tudi prisotni študentje. Nastja je povedala, da si študentje želimo študija v živo, hkrati pa podpiramo, da bi se predavanja še vedno snemala/prenašala ali pa da bi se objavljali lanski posnetki, tako da bi imeli tudi študentje, ki so prisiljeni ostati doma (bodisi zaradi svoje bolezni ali pa bolezni nekoga iz gospodinjstva), priložnost slediti predavanjem.
    
    \item
    Nastja povzame dogajanje v ZPS Oddelka za fiziko: govora je bilo o poteku študija v prihajajočem študijskem letu in o organizaciji pedagoške konference, na katero smo bili povabljeni tudi predstavniki ŠS FMF. Drugih posebnosti ni bilo.
    
    Gašper pove, da se na sejah ZPS Oddelka za matematiko ni zgodilo nič posebnega.
    
    Matej povzame dogajanje na Senatu FMF: pogovarjali so se o poteku študija.
    
    Katarina povzame dogajanje v Komisiji za študijske zadeve: letos je bilo zaradi epidemije Covid-19 ponovno več prošenj za podaljšanje statusa študenta, ki so jim v veliki meri ugodili (če so le prosilci prošnjo argumentirali).
    
    Matic povzame dogajanje v Disciplinski komisiji: izvedel se bo postopek proti enemu študentu (več podrobnosti zaradi občutljive narave informacij ni bilo moč podati).
    \item 
    \begin{sklep*}
    Svet sprejme študentska mnenja o naslednjih pedagoških delavcih:
        \begin{itemize}
            \item Jernej Fesel Kamenik
            \item Žiga Gregorin
        \end{itemize}
    Svet zaradi pomanjkanja informacij ne izda mnenja o kandidatu Žigi Zaplotniku.
    \end{sklep*}

    \item
    \begin{sklep*}
        \begin{itemize}
            \item Matej Janežič odstopi z mesta predsednika ŠS FMF.    
        \end{itemize}
    \end{sklep*}
    
    \begin{sklep*}
        \begin{itemize}
            \item ŠS FMF imenuje Nastjo Zupančič za predsednico ŠS FMF.
            \item ŠS FMF imenuje Gašperja Žajdelo za podpredsednika ŠS FMF.
        \end{itemize}
    \end{sklep*}
    
    \item
    Predsednica ŠS FMF seznani ostale člane o tem, da je ŠS UL sprejel novi Pravilnik o volitvah predstavnikov študentov v študentske svete članic in organe članic Univerze v Ljubljani. V skladu z njim bo potrebno spremeniti Poslovnik ŠS FMF, katerega nova različica bo obravnavana na prihajajoči dopisni seji. 
    
    \item
    Gašper pove, da se je nanj obrnil dr. Matija Pretnar in sicer glede Nagrade Marjana Jermana. Zanimalo ga je, kakšno je mnenje ŠS FMF glede dodajanja primera vloženega predloga za nagrado k splošnemu obvestilu študentom. ŠS FMF doreče, da naj se za začetek pošlje študentom obvestilo brez dodanega primera. Če bo prišlo do večjega povpraševanja študentov po tem, kako naj bo predlog za nagrado oblikovan, se naknadno pošlje še izpolnjen primer.
    
    ŠS FMF se pogovarja o sodelovanju na sprejemu brucev v petek, 1. 10. 2021. Ker noben od članov na tisti dan ne bo prisoten na fakulteti, se dogovorimo, da bo Nastja drugi teden oktobra za prve letnike naredila kratko predstavitev ŠS FMF (o specifičnih urah in datumih predstavitve za posamezne študijske programe/smeri se bo naknadno  dogovarjala s posamičnimi profesorji).
    
\end{ad}


\end{document}