\documentclass{seja}

\newcommand{\SejaNum}{2}
\newcommand{\Title}{Zapisnik \SejaNum.~redne seje Študentskega sveta FMF}
\newcommand{\DateSeja}{8.~4.~2021}
\newcommand{\DateZapisnik}{8.~4.~2021}

\begin{document}

Prisotni člani in članice ŠS FMF UL:
Matej Janežič, Matevž Kopač, Žan Mikola, Aljoša Rebolj, Katarina Šipec, Klara Širca, Matic Tonin, Tjaša Vagaja, Jakob Zmrzlikar, Nastja Zupančič, Gašper Žajdela.

Opravičeno odsotni člani in članice ŠS FMF UL:
Nina Velkavrh

\begin{red*}
	\item
	Sprejem zapisnika prejšnje seje
	\item
	Poročilo predsedstva
	\item
	Poročilo predstavnikov v organih in komisijah
    \item
    Sprejem študentskih mnenj
    \item 
    Nagrada Marjana Jermana za najboljšega asistenta
    \item
    Mafijski piknik
    \item
    Spletna stran ŠS FMF
\end{red*}

\begin{ad}
    \item
    \begin{sklep*}
        ŠS FMF sprejme zapisnik prejšnje seje.
    \end{sklep*}
    
    \item
    Predsednik ŠS FMF povzame, da je študentski svet uspešno razdelil vse puloverje, zahvali se Gašperju za pomoč pri organizaciji ponudbe za puloverje. ŠS je uspešno v kratkem času organiziral tekmovanje v recitiranju števila pi. Izpostavi, da so imeli pri organizaciji pomoč iz veliko delov fakultete. Medijsko je bil dogodek dobro pokrit, predvsem zaradi rekorderja Tiborja.
   
    \item
    Gašper povzame dogodke v ZPS oddelek matematika: minimalne spremembe v sklopu predmeta statistika v 3. letniku, pol ure dodatne na teden. Upokojil se je profesor Petkovšek, nekatere predmete zdaj vodi profesor Pretnar. Na programu uporabna statistika so se spremenili pogoji za vpis; sedaj bo za vpis upoštevan tudi sprejemni izpit. Glede omejitve vpisa na programe ni sprememb.
   
    Nastja povzame dogodke v ZPS oddelek fizika: Predlagali preimenovanje nekega predmeta ter uvedli minimalne spremembe glede samoevaluacijskih poročil. Govora je bilo tudi o izmeničnem predlaganju asistenta za nagrado Marjana Jermana; enkrat fiziki, drugič matematiki.
   
    Matevž povzame dogajanje v UO: sprejeti so bili računovodski izkazi.
  
    Matej povzame dogajanje v senatu: pregledalo se je poslovno in računovodsko poročilo ter poročilo o samoevaluaciji, večino pa se je govorilo o kandidaturi dekana za rektorja.

    Seja ŠS UL: govorilo se je o obvezni športni vzgoji, večina ŠS UL pa je bila proti. V sestavi je tudi enoten obrazec za pisanje študentskih mnenj po vseh fakultetah.
 
    \item
    \begin{sklep*}
        Svet sprejme študentska mnenja o naslednjih pedagoških delavcih:
        \begin{itemize}
            \item Aleksey Kostenko
            \item Grega Podlogar
            \item Gregor Šega
            \item Lara Vukšić
        \end{itemize}
            
        Svet zaradi pomanjkanja informacij ne izda mnenja o kandidatki Nataši Čelan Korošin.
    \end{sklep*}
   
    \item
    ŠS FMF govori o tem, da je bil kontaktiran s strani profesorja Pretnarja glede nagrade Marjana Jermana za najboljšega asistenta. Nagrada se lahko podeli vsako leto, ni pa obvezno. Komisijo za izbor sestavlja 6 članov: 4 študenti ter 2 profesorja. Kandidat je lahko kdorkoli, ki je tisto leto vodil vaje na FMF. Denarna nagrada je 1000 EUR bruto. Nagrada naj bi se delila izmenično: 1 leto za matematike in 1 za fizike, vsakih 5 let pa je asistent lahko nagrajen. Kandidata lahko predlaga kdorkoli. V planu je, da se že letos organizira podelitev nagrade, vendar pa mora biti pravilnik sprejet prvo na senatu. ŠS FMF določi komisijo. 
    
    \item
    Mafijskega pikinika letos ne bo v standardni obliki. ŠS govori o ideji, da bi lahko maja vsak na socialna omrežja delil slike, kako ima na vrtu svoj piknik. Tako bi se ustvarilo neko dobro vzdušje, za dobre slike pa bi ponudili tudi neke nagrade in organizirale kakšen večer družabnih iger. 
    ŠS FMF ustvari delovno skupino za organizacijo majskih dogodkov v sestavi Jakob Zmrzlikar, Matej Janežič, Klara Širca, Nina Velkavrh, Gašper Žajdela.
    
   
    
    \item
    ŠS FMF ustvari delovno skupino za urejanje spletne strani v sestavi: Matej Janežič, Katarina Šipec, Matic Tonin, Jakob Zmrzlikar, Nastja Zupančič.
    \item
    ŠS FMF se pogovari o vlogi predstavnika ŠS FMF, da je pomembno, da vsak član poroča o vsem, kar se dogaja v njegovi komisiji. 
    Prav tako se je ŠS FMF pogovoril o tem, da se je javil študent, ki želi predati delo urejanja Študentskih strani FMF drugemu študentu. Dobro bi bilo narediti razpis in ga objaviti med študente.
    ŠS FMF na dan volitev za rektorja UL spomni ostale študente na socialnih omrežjih, da so volitve.
    
    
    
    
    
    
    
    
    
    
    
    
    
    
    
    
    
    
    
    
    
\end{ad}
\end{document}