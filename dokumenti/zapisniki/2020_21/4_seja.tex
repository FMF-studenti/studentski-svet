\documentclass{seja}

\newcommand{\SejaNum}{4}
\newcommand{\Title}{Zapisnik \SejaNum.~redne seje Študentskega sveta FMF}
\newcommand{\DateSeja}{9.~6.~2021}
\newcommand{\DateZapisnik}{9.~6.~2021}

\begin{document}

Prisotni člani in članice ŠS FMF UL:
Matej Janežič, Žan Mikola, Katarina Šipec, Klara Širca, Matic Tonin, Tjaša Vagaja, Nina Velkavrh, Jakob Zmrzlikar, Nastja Zupančič, Gašper Žajdela.

Opravičeno odsotni člani in članice ŠS FMF UL: Klemen Ducman, Matevž Kopač, Aljoša Rebolj.

\begin{red*}
	\item
	Sprejem zapisnika prejšnje seje
	\item
	Poročilo predsedstva
	\item
	Poročilo predstavnikov v organih in komisijah
    \item
    Sprejem študentskih mnenj
    \item
     Razno
\end{red*}

\begin{ad}
    \item
    \begin{sklep*}
        ŠS FMF sprejme zapisnik prejšnje seje.
    \end{sklep*}

\item 
Matej povzame, da ga je kontaktiral doc. dr. Cigler, v kolikor bi lahko našel študenta, ki bi povedal nekaj besed na žalni seji za doc. dr. Marjana Jermana. Povedal je tudi, da išče, ali obstajajo kakšne slike doc dr. Jermana s študenti.
\item 
Gašper povzame dogodke v ZPS oddelek matematika: govorilo se je o neustreznih prevodih doktorskih nalog ter o tem, da so spremenili pogoje za napredovanje na programu finančna matematika. Dotična sprememba je le sprememba imena predmeta.

Nastja povzame dogodke v ZPS oddelek fizika: govorilo se je o tem, ali bi poslikali fasado Fakultete za fiziko.

Matej povzame dogajanje v senatu: v senatu so sprejeli pravilnik za podeljevanje nagrade Marjana Jermana ter govorili o volitvah dekana. Sedaj je naš dekan dr. Ramšak postal prorektor univerze, zato smo imeli nove volitve za dekana in dobili novega dekana dr. Tomaža Koširja.
    
Gašper in Nastja povzameta dogajanje na seji ŠS UL: imeli so nove volitve za nekatere pozicije, sprejeti pa so bili enotni obrazci za študentska mnenja. So se pogovarjali tudi o tem, ali bi pisci študentskih mnenj enotno bili anonimni ali ne. Nastja doda, da se je govorilo o tem, da bi ŠS FMF sodeloval pri promociji cepljenja, ne bi pa bil nosilec projekta.

   \begin{sklep*}
        ŠS FMF se odloči, da sodeluje pri projektu promocije cepljenja ŠS UL, ni pa nosilec projekta.
    \end{sklep*}

\item
\begin{sklep*}
        Svet sprejme študentska mnenja o naslednjih pedagoških delavcih:
        \begin{itemize}
            \item Gregor Traven
            \item Simon Čopar
            \item Luka Šantelj
            \item Jakob Novak
            \item Stanislav Vrtnik
            \item Janoš Vidali
            
        \end{itemize}
            
        Svet zaradi pomanjkanja informacij ne izda mnenja o naslednjih kandidatih: Boštjanu Končarju ter Mitji Uršiču.
    \end{sklep*}

\item
Nastja izpostavi, da lahko pridobi prosojnice o GDPR ter varovanju podatkov, v kolikor bi morali delovanje ŠS FMF prilagoditi GDPR pravilniku.

V ŠS FMF se pogovorimo o tem, da bi dali izjavo glede zakona o tujcih in potrebnega dokazila o finančnih sredstvih za tuje študente. Dogovorimo se, da počakamo na naslednjo sejo ŠS UL, nato pa na podlagi mnenja, ki si ga ustvarimo tam, tudi mi podamo izjavo glede te teme.
\end{ad}
\end{document}