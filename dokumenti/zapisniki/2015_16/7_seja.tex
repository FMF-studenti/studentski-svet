\documentclass{seja}

\newcommand{\SejaNum}{7}
\newcommand{\Title}{Zapisnik 2.~dopisne (\SejaNum.~redne) seje Študentskega sveta FMF}
\newcommand{\DateSeja}{18.~3.~2016 -- 21.~3.~2016}
\newcommand{\DateZapisnik}{18.~4.~2016}

\begin{document}
Prisotni člani in članice ŠS FMF UL:
Rok Gregorič, Matej Jazbec, Bruno Jurič, Veno Mramor, Aljaž
Robek, Jure Slak, Ana rebeka Kamšek, Ana Smerdu in Marion van Midden

Opravičeno odsotni člani in članice ŠS FMF UL: Ana Rebeka Kamšek

\begin{red}
\item Predlog ZPS OM, da bi na 1. stopnji uvedli vnaprej določene datume rednih pisnih izpitov
\item Potrditev pripomb in predlogov v zvezi z novelo ZViS-a
\item Dopis glede pridobitve certifikata LGBT prijazno
\end{red}
Dnevni red soglasno potrjen.

\begin{ad}

\item Kot je bilo dogovorjeno na 6. rednji seji, je Svet oblikoval predlog dopisa z naslednjim besedilom.

\emph{ŠS FMF se je na 5.~redni seji, 10.~3.~2016 seznanil s predlogom spremembe ureditve izpitnih rokov na Oddelku za matematiko. Zaradi uskladitve s statutom ZPS OM predlaga, da bi bili vsi redni pisni izpitni roki na oddelku vnaprej določeni.}

\emph{V ŠS FMF smo predlogu v splošnem naklonjeni, saj takšna ureditev omogoča dobro razporeditev
izpitov in posledično ugodnosti tako za asistente in profesorje, kot tudi za študente. Pri tem je treba opozoriti na to, da bi bilo potrebno pri načrtovanju izpitov paziti, da se tudi študentom, ki določene predmete ponavljajo, roki ne bi prekrivali. Hkrati nas skrbi, da bi vnaprejšnje določanje vseh rokov zmanjšalo fleksibilnost študija. Predvsem je ta pomembna pri 3.~rokih, ki so tipično konec avgusta ali septembra in jih piše majhno število študentov. Menimo, da teh rokov ni smiselno fiksirati, ker bi se s tem po nepotrebnem oteževalo študijski proces.}

\emph{Tudi pri prvih dveh rokih predlagamo, da se hkrati z morebitno uvedbo vnaprejšnjega določanja
sprejme tudi natančno opredeljen formalen postopek s katerim se lahko ob eksplicitno izraženem
nezadovoljstvu nekega smiselno vnaprej določenega deleža študentov rok izpita vseeno prestavi, v kolikor je to mogoče in se s tem strinja asistent/profesor pri predmetu.}

\emph{Naposled predlagamo, da se omenjene spremembe navezujejo zgolj na redne izpitne roke na 1.~stopnji študija, na 2.~stopnji pa naj zaradi majhnega števila študentov in velike količine izbirnih predmetov ostane dosedanji izpitni režim.}

\emph{Upamo, da bo ta sprememba, če se oddelek zanjo odloči, izpeljana natančno in previdno ter, da bo pozitivno doprinesla k delovanju oddelka in študiju matematike na FMF.}

\begin{sklep*}
ŠS FMF sprejme sklep, da se dopis pošlje vodstvu oddelka za matematiko
\end{sklep*}

\item Študentski svet je glede predlagane novele ZViS-a ugotovil naslednje:

Študentski svet FMF bi glede novele zakona o visokem šolstvu izpostavil naslednje pomanjkljivosti
in pripombe:
\begin{enumerate*}
\item V zakonu bi morala biti jasno izražena delitev med rednim in izrednim študijem in točno opredeljene
in zamejene javne in zasebne dejavnosti.
\item Pogoji glede izvajanja programov v tujem jeziku bi morali biti bolj natančni in restriktivni.
Z izvajanjem programov v tujem jeziku je omejena dostopnost brezplačnega študija za slovenske
državljane.
\item Vprašljivost smiselnosti enkratne akreditacije programov . Posledično bodo imele fakultete bolj
proste roke in ker bodo morale konkurirati na trgu, bodo posledično dajale poudarek le na aplikativna
področja in lahko zapostavljala ali celo ukinjala tista manj dobičkonosna, teoretična področja.
\item Kriteriji za ustanavljanje samostojnih šolskih zavodov niso zadovoljivi.
\item Problematičnost zaračunavanja šolnin študentom, ki že imajo pridobljeno določeno stopnjo izobrazbe.
Tako študentom, ki so pridobili določeno stopnjo izobrazbe v tujini v Sloveniji nimajo
dostopa do brezplačnega študija na prvi in drugi stopnji (oz. tem ekvivalentnim programom). Prav tako ne vidimo smisla, zakaj bi sposobne študente, ki želijo študirati vzporedno ali zaporedno omejevali in jim s šolninami preprečevali oz. oteževali dodatno izobraževanje.
\end{enumerate*}

\begin{sklep*}
ŠS FMF posreduje zgoraj navedene pripombe pristojnim.
\end{sklep*}

\item Svet se seznani s predlogom dopisa.
\begin{sklep*}
ŠS FMF bo vodstvu fakultete posredoval dopis v katerem bo vodtsvo pozval k pridobitvi
certifikata LGBT prijazna.
\end{sklep*}
\end{ad}

% spremeni footer
\makeatletter \global\let\@enddocumenthook\@empty \makeatother
\AtEndDocument{
Zapisala Marion van Midden, pregledal Jure Slak.

\begin{flushright}
  Predsednica ŠS FMF \\
  Marion van Midden
\end{flushright}}

\end{document}
