\documentclass{seja}

\newcommand{\SejaNum}{2}
\newcommand{\Title}{Zapisnik \SejaNum.~redne seje Študentskega sveta FMF}
\newcommand{\DateSeja}{24.~11.~2015}
\newcommand{\DateZapisnik}{13.~12.~2015}

\begin{document}

Prisotni člani in članice ŠS FMF UL:  Rok Gregorič, Matej Jazbec, Bruno Jurič,
Ana Rebeka Kamšek, Veno Mramor, Aljaž Robek, Jure Slak, Ana Smerdu in Marion van
Midden

\begin{red*}
\item Sprejem zapisnika 1.~seje
\item Študentske ankete
\item Opombe predstojnice oddelka za matematiko
\item Študentska mnenja
\item Poslovnik ŠS
\item Amandmaji k statutu UL
\end{red*}

\begin{ad}
\item
\begin{sklep*}
  ŠS FMF potrjuje zapisnik 1.~seje.
\end{sklep*}

\item
Svet se seznani z novim sistemom anket. Zavzame se, da bo poskrbel za dobro informiranost
študentov o spremembi in delovanju novega sistema. Ko bo na voljo več informacij, bo Svet poskušal iskati načine kako motivirati študente, da bi reševali ankete.

\item
Svet je seznanjen z željo predstojnice oddelka za matematiko in njenega namestnika po
sestanku z matematičnim delom sveta. V drugem in tretjem nadstropju stavbe Fakultete za
Matematiko so študenti v straniščih dobili papirnate brisače. Predstojnica oddelka za
matematiko poudarja, naj Svet skrbi za higieno in da se ne smeti. Svet se zaveže, da bo
na napravo s papirnatimi brisačami nalepil obvestilo. Predstojnica še poudarja, da mora
svet informirati študentsko javnost. V ta namen so mu na voljo elektronske table na
fiziki. Svet se zaveže, da bo objavljal vabila na seje tudi na te table, da so na
voljo javnosti.

\item
Svet je seznanjen z mnenji.

\begin{sklep*} ŠS FMF sprejme pozitivno mnenje o Martinu Raiču.        \end{sklep*}
\begin{sklep*} ŠS FMF sprejme pozitivno mnenje o Niku Stoparju.        \end{sklep*}
\begin{sklep*} ŠS FMF sprejme pozitivno mnenje o Krisu Stoparju.       \end{sklep*}
\begin{sklep*} ŠS FMF sprejme pozitivno mnenje o Leonu Cizlju.         \end{sklep*}
\begin{sklep*} ŠS FMF sprejme pozitivno mnenje o Urbanu Jezerniku.     \end{sklep*}
\begin{sklep*} ŠS FMF sprejme negativno mnenje o Risteju škrekovskem.  \end{sklep*}

\textit{Bruno Jurič zapusti sejo.}

\item
Svet ustanovi delovno telo za posodobitev poslovnika.
DT bo poročalo o svojem delu na naslednji seji.

\item Svet je seznanjen z nekaterimi potencialno spornimi točkami predlaganega statuta UL.

\begin{sklep*}
ŠS FMF predlaga amandma k 79.~členu statuta, da se mandat ŠS članice
iz dveh let skrajša na eno leto.
\end{sklep*}

\begin{sklep*}
ŠS FMF predlaga amandma k 89.~členu statuta, da se poleg v
slovenskem predavanja sme izvajati tudi v tujem jeziku.
\end{sklep*}

\begin{sklep*}
ŠS FMF predlaga amandma k 26.~členu statuta,
da se iz besedila člena izvzame besedo živorojenega.
\end{sklep*}

\begin{sklep*}
ŠS FMF predlaga amandma k 93.~členu statuta,
da se status študenta konča z začetkom naslednjega študijskega leta.
\end{sklep*}

\begin{sklep}
ŠS FMF predlaga amandma k 120.~členu statuta,
da se pri predmetih, ki imajo tudi vaje, lahko izvajalec odloči,
da bo predmet imel dve oceni.
\end{sklep}
Sklep je sprejet s 7 glasovi za, 1 glasom prosti in 0 vzdržanih.

\textit{Ana Smerdu zapusti sejo.}

\begin{sklep*}
ŠS FMF predlaga amandma k četrtemu odstavku 137.~členu statuta,
da se besedilo iz ``oblikuje'' spremeni v ``svetuje v oblikovanju''.
\end{sklep*}

\item
Svet bo javnosti posredoval obvestilo o šahu. Predsednica bo kontaktirala
BEST in poskušala izvedeti več o njihovi prošnji.

Naslednja seja bo izjemoma v torek 15.~12.~ob 14ih.

\end{ad}

\end{document}
