\documentclass{seja}

\newcommand{\SejaNum}{4}
\newcommand{\Title}{Zapisnik \SejaNum.~redne seje Študentskega sveta FMF}
\newcommand{\DateSeja}{27.~2.~2020}
\newcommand{\DateZapisnik}{7.~4.~2020}

\begin{document}

Prisotni člani in članice ŠS FMF UL:
Matej Janežič,
Timotej Lemut (od začetka do konca razprave o podjetniškem klubu),
Klementina Pirc,
Aljoša Rebolj,
Katarina Šipec,
Luka Umek,
Jakob Zmrzlikar,
Nastja Zupančič (do konca razprave o Pi dnevu),
Gašper Žajdela.
\\
Opravičeno odsotni člani in članice ŠS FMF UL: Jan Krneta, Luka Školč.

\begin{red*}
	\item
	Sprejem zapisnika prejšnje seje
	\item
	Poročilo predsedstva
	\item
	Poročilo predstavnikov v organih
    \item
    Sprememba poslovnika KŠM
    \item
    Podjetniški klub 
    \item
    Pi dan
    \item
    Mafijski piknik

\end{red*}

\begin{ad}
    \item
    \begin{sklep*}
        ŠS FMF sprejme zapisnik prejšnje seje.
    \end{sklep*}
    
    \item
    Poročilo predsedstva: 
    \begin{itemize}
        \item Razdeljevanje Hoodijev je bilo uspešno izpeljano in je potekala brez zapletov.
        \item Fakulteta je bila dobro zastopana na Informativi in tudi na informativnih dnevih.
        \item Pojavilo se je veliko idej za nove promocijske izdelke fakultete: čokoladke z motivom FMF ter skodelice z matematičnim motivom, saj skodelice s fizikalnim motivom že obstajajo. ŠS bo v sodelovanju z vodstvom fakultete izbral primeren matematični motiv.
    \end{itemize}
    
    \item
    Poročilo predstavnikov v organih: 
    
    Predstavnik v Senatu poroča o preteklih treh rednih sejah. 
    \begin{itemize}
        \item Senat je sprejel poslovno poročilo in finančni načrt.
    \end{itemize}
    
    Predstavnik UO FMF poroča o pretekli seji:
    % tukile vejice niso okej :") bo Katarina pokomentirala :P
    \begin{itemize}
        \item Finančno načrtovanje poteka v treh stopnjah: najprej se pripravi finančni načrt, nato se ga realizira, na koncu pa sledi še ocena realizacije. Na seji so obravnavali vse tri stopnje načrtovanja po vrsti za leta 2019, 2018 in 2017. Finančni načrt za leto 2020 bi morali že predstaviti, a je bil odbor odpovedan. 
        \item Sprejet je bil sklep, da so demonstranti, ki so študenti enovitega študija pedagoške matematike in imajo opravljene vse obveznosti prvih treh letnikov, plačani enako kot magistrski študenti. 
    \end{itemize}
    
    Predstavnica v ZPS OM poroča o preteklih dveh sejah:
    \begin{itemize}
        \item Januarska seja: za študente, ki hkrati opravljajo programa MAT-2 in PEM-2 je bil predlagan sklep, da se lahko naredi samo eno magistrsko delo namesto dveh.
        \item Februarska seja: sprejetih je bilo veliko sprememb študijskih programov, prav tako pa je prišlo do spremembe obveznih predmetov.
    \end{itemize}
    
    Predstavnica v ZPS OF poroča o preteklih sejah:
    \begin{itemize}
        \item Višješolski program Fizika Merilna Tehnika je bil preimenovan v Aplikativna Fizika. 
        \item Spremenjen je bil tudi predmetnik tega programa, tako da je zdaj bolj aktualen.
    \end{itemize}
    
    \item
    \begin{sklep*}
    Svet razreši Katarino Šipec in Nastjo Zupančič z mesta predstavnikov v ŠS UL.
    \end{sklep*}
    
    Svet se seznani z dejstvom, da sta Matej Janežič in Klementina Pirc predstavnika v ŠS UL.
    \begin{sklep*}
    Svet izvoli Katarino Šipec in Nastjo Zupančič za namestnici v ŠS UL.
    \end{sklep*}
    
    \item Svet obravnava predlagano spremembo poslovnika KŠM. Spremenijo se 2., 3., 6. in 7. člen. 
    
    Timotej Lemut pride na sejo.
    
    \item
    Prvega sestanka podjetniškega kluba se je udeležilo veliko ljudi. Kmalu bo organiziran tudi drugi sestanek. Če bo interesa dovolj, bodo udeleženci do konca leta poskušali najti način, da bi se klub obdržal tudi v naslednjih letih.
    
    Timotej Lemut zapusti sejo.
    
    \item Pi dan: 
    Zaradi higienskih razlogov in novega koronavirusa letos na Pi dnevu ne bo pit. V primeru dodatnih ostrejših ukrepov zaradi koronavirusa se bo datum dogodka premaknil. Pripravljene nagrade so: posebni termometri za prva tri mesta in druge tolažilne nagrade vse do 10. mesta.
    
    Nastja Zupančič zapusti sejo.
    
    \begin{sklep*}
    Ustanovi se delovna skupina za organizacijo Pi daneva, v kateri so: Klementina Pirc, Matej Janežič, Katarina Šipec in Luka Umek.
    \end{sklep*}
    
    \item Mafijski piknik: 
    Organizira se natečaj za oblikovanje majic za Mafijski piknik. 
    
\end{ad}

\end{document}